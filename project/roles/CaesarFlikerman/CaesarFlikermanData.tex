\newsubsection{Цезарь Фликерман --- Чудаков Виталий}{Цезарь Фликерман}

\begin{itemize}
\item \textbf{Действия в течение игры.} Интервью участников, жителей Капитолия. Проведение визиток, интервьюирование  групп, находящихся непосредственно на <<сцене>>. Освещение мероприятия в целом.

\item \textbf{Дистрикт №1. ИВТ-11}\\
Наш первый участник – дистрикт № 1 (ИВТ-11) – самый богатый, самый успешный, любимчики Капитолия. Этот дистрикт славится своими ювелирами и бойцами. Их украшения носит весь Капитолий. Говорят, сам Президент Сноу на званые вечера не брезгует изделиями этого дистрикта. Бойцы дистрикта № 1 тренируются с малых лет, готовятся стать трибутами. Это опасные противники.

\item \textbf{Дистрикт №4. ИВТ-12}\\
Наш следующий участник – дистрикт № 4 (ИВТ-12) – наши незаменимые рыболовы. Вы хотите устриц или кальмаров? Или вас интересуют креветки? Эту проблему может решить практически каждый житель четвертого дистрикта. Они в совершенстве владеют холодным оружием и могут достать еду из любого водоема. Как и в первом, трибуты четвертого дистрикта тренируются с детских лет и вызываются на ИГРЫ добровольцами. Хотите выжить – прячьте от них ножи.

\item \textbf{Дистрикт №5. ИТ-11}\\
Третий участник наших ИГР – дистрикт № 5 (ИТ-11). Ночью Вы идете по улицам Капитолия и не замечаете темноты? У вас дома работают телевизор и холодильник? У Вас нет проблем с тем, чтобы разогреть себе еду? Все это благодаря жителям дистрикта № 5. Электрическая, солнечная, ядерная - дистрикт № 5 использует энергию земли и неба, чтобы дать силу нашему великому государству. Большинство трибутов из этого дистрикта – неплохие инженеры. Из пары микросхем и мотка проволоки они способны создать шедевр технической мысли.

\item \textbf{Дистрикт №5. ПИЭ-11}\\
Представляем очередного участника. Это дистрикт № 7 (ПИЭ-11). Наши доблестные дровосеки, столяры и плотники. Все деревянное, что есть вокруг Вас, сделали они. К сожалению, в Капитолии предпочитают металл, поэтому дискрикт № 7 не может похвастаться роскошью и шиком. Там живут обычные трудяги, и никакой подготовки к ИГРАМ у них нет. Но не стоит их недооценивать – топорами они владеют превосходно.

\item \textbf{Дистрикт №5. ИТ-12}\\
Последний (но не по значению!) участник – дистрикт № 8 (ИТ-12). Модельеров и стилистов здесь нет. Они просто производят ткани и шьют одежду. От простых и прекрасных материалов для жителей дистриктов до великолепной одежды из парчи для Капитолия - всё это создаёт дистрикт № 8. Да, они не богаты. Да, они не тренируются перед ИГРАМИ. Но они привыкли жить скромно, в спартанских условиях, и умеют работать руками. А иногда именно это может спасти Вам жизнь.
\end{itemize}