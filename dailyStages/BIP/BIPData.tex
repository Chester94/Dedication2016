\newsubsection{БИП --- Чехранов Сергей}{БИП}

\par Этап заключается в преодолении некой области на земле. На земле сигнальной лентой создается сетка 6x6. Организатор имеет у себя <<карту>> этой сетки. На сетке есть безопасные участки и участки, на которые нельзя вступать. Существует маршрут по соседним клеткам каждая из которых безопасна из левой нижней в верхнюю правую. Клетки считаются соседними, если они соприкасаются стороной. По диагонали --- не соседи.

\par Задача команды \textbf{без слов, молча} найти это маршрут. Если участник во время <<прощупывания>> маршрута <<подорвался>>, то он возвращается к команде и \textbf{другой} участник команды продолжает поиск. В итоге, если маршрут найден все <<живые>> участники проходят по маршруту самостоятельно. <<Подорвавшихся>> переносят на спине <<живые>>.

\par Обо всех правилах заранее сообщается во вступительной речи \textbf{проводящего этап}.

\par \textbf{Речь для объяснения правил.} В разработке