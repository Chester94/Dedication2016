\newsection{Отдельные моменты концепции}{Отдельные моменты концепции}

\begin{itemize}
\item Факультету ИВТ в этом году (2016) исполняется 30 лет. В связи с этим мы говорим, что проводятся \textbf{30-е, юбилейные <<Голодные игры>>}.

\item Так как игры юбилейные, то и правила отличаются от оригинальных. От каждого дистрикта приезжает не 2 участника, а целая команда.

\item Организаторы, которые не являются кураторами (менторами), считаются жителями Капитолия. Необходимо соответствовать этому статусу (в разумных пределах).

\item Профорг и его заместитель, как лица наиболее ответственные, могут не отыгрывать роли жителей Капитолия, так как их первоочередная цель --- обеспечение соблюдения правил проведения мероприятия.

\item Предполагается создание деталей костюма <<жителей Капитолия>>, для большей схожести с прототипами. Достаточно вызывающие, яркие, необычные детали одежды и/или макияжа. \textbf{Данный пункт требует отдельного обсуждения}.

\item В процессе всего мероприятия группы набирают баллы (или иные эквивалентные им единицы). Баллы начисляются за лучший лагерь, лучшую визитку\footnote{см. раздел <<Визитки команд>>}, за успешное выполнение этапа дневной части\footnote{см. раздел <<Дневные этапы>>}, за успехи в спортивных мероприятиях\footnote{см. раздел <<Спортивные мероприятия>>}, за успешное прохождение этапов ночной игры\footnote{см. раздел <<Ночная игра>>}. Возможно добавление иных пунктов получения баллов.

\item Было бы замечательно донести до первокурсников тематику посвята до начала посвята.

\item <<ИВТ сегодня, ИВТ завтра, ИВТ всегда>>.

\item <<Счастливых вам <<Голодных игр>>, и пусть удача всегда будет на вашей стороне>>.
\end{itemize}