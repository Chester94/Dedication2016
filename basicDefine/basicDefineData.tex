\newsection{Основные понятия}{Основные понятия}

\begin{itemize}
\item \textbf{Панем} --- страна на территории нынешних Соединенных Штатов Америки. Именно в Панеме разворачиваются события романа <<Голодные игры>>\footnote{см. раздел <<Основная сюжетная линия>>}. В нашем случае Панем заменяется на факультет ИВТ.

\item \textbf{Капитолий} --- столица Панема, населенная избалованными жителями, излюбленным развлечением которых является просмотр жестоких сражений между подростками, столица страны. В роли Капитолия выступает лагерь организаторов, сами же организаторы представляют население города (фактически зрителей <<Голодных игр>>).

\item \textbf{Дистрикт} --- один из 12(13) районов Панема, находящийся под контролем Капитолия. Специализируется на конкретной области промышленности.

\item \textbf{Трибут} --- участник игр от конкретного дистрикта.

\item \textbf{Ментор} --- победитель одной из прошлых игр. Наставник трибутов. В нашем случае менторы -- это кураторы групп.
\end{itemize}