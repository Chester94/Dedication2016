\documentclass[a4paper, 12pt]{extarticle}
\usepackage[english, russian]{babel}
\usepackage[utf8x]{inputenc}
\usepackage{fullpage}
\usepackage{indentfirst} % Первый абзац в разделе тоже с красной строки
\usepackage{cmap} % для кодировки шрифтов в pdf (чтобы не было крокозябры при копировании из pdf )
\usepackage{graphicx} % для вставки картинок
\sloppy % Включение переноса слов в тексте

\usepackage{hyperref} % Для добавления ссылок в тесте

% Настройка цветов для ссылок
\hypersetup{
colorlinks = true,
linkcolor = black,
pagecolor = black,
urlcolor = blue, 
citecolor = black
}

\usepackage[labelfont=bf, labelsep=space]{caption} % Делаем надписи "Рис.1" под рисунками жирными и без двоеточия.

\usepackage[top=20mm, bottom=20mm, left=20mm, right=20mm
, nohead % Убрать расстояние для верхних колонтикулов
%, nofoot % Убрать расстояние для нижних колонтикулов
]
{geometry} % Размер полей у старницы
\setlength{\parindent}{1.25cm} % Размер интервала для абзацев 
\usepackage{setspace}
\onehalfspacing % одинарный интервал

\usepackage{caption} % подписи к рисункам в русской типографской традиции
\DeclareCaptionFormat{GOSTtable}{#2#1\\#3\vspace*{-\baselineskip}}
\DeclareCaptionLabelSeparator{fill}{\hfill}
\DeclareCaptionLabelFormat{fullparents}{\bothIfFirst{#1}{~}#2}
\captionsetup[table]{
     format=GOSTtable,
     %font={footnotesize},
     labelformat=fullparents,
     labelsep=fill,
     labelfont=normal,
     textfont=bf,
     justification=centering,
     singlelinecheck=false
     }

\begin{document}

\begin{figure}[h]
	\center{
	\includegraphics[width=0.49\linewidth]{d1.png}
	\includegraphics[width=0.49\linewidth]{d2.png}
	}
\end{figure}

\begin{figure}[h]
	\center{
	\includegraphics[width=0.49\linewidth]{d3.png}
	\includegraphics[width=0.49\linewidth]{d4.png}
	}
\end{figure}

\begin{figure}[h]
	\center{
	\includegraphics[width=0.49\linewidth]{d5.png}
	\includegraphics[width=0.49\linewidth]{d7.png}
	}
\end{figure}

\begin{figure}[h]
	\center{
	\includegraphics[width=0.49\linewidth]{d8.png}
	\includegraphics[width=0.49\linewidth]{d10.png}
	}
\end{figure}

\begin{figure}[h]
	\center{
	\includegraphics[width=0.49\linewidth]{d11.png}
	\includegraphics[width=0.49\linewidth]{d12.png}
	}
\end{figure}

\end{document}