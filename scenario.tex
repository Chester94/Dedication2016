\documentclass[a4paper, 14pt]{extarticle}
\usepackage[english, russian]{babel}
\usepackage[utf8x]{inputenc}
\usepackage{fullpage}
\usepackage{indentfirst} % Первый абзац в разделе тоже с красной строки
%\usepackage{cmap} % для кодировки шрифтов в pdf (чтобы не было крокозябры при копировании из pdf )
\usepackage{graphicx} % для вставки картинок
\graphicspath{{res/}} % Путь к папке с картинками
\sloppy % Включение переноса слов в тексте

\title{Сценарий культурно--массового мероприятия <<Посвящение первокурсников 2016>>}
\author{Культурно--массовая комиссия факультета ИВТ}

\usepackage{color}

\usepackage{hyperref} % Для добавления ссылок в тесте

% Настройка цветов для ссылок
\hypersetup{
colorlinks = true,
linkcolor = black,
pagecolor = black,
urlcolor = blue, 
citecolor = black
}

\usepackage[labelfont=bf, labelsep=space]{caption} % Делаем надписи "Рис.1" под рисунками жирными и без двоеточия.

\usepackage[top=20mm, bottom=20mm, left=30mm, right=20mm
, nohead % Убрать расстояние для верхних колонтикулов
%, nofoot % Убрать расстояние для нижних колонтикулов
]
{geometry} % Размер полей у старницы

\setlength{\parindent}{1.25cm} % Размер интервала для абзацев 
\usepackage{setspace}
\onehalfspacing
\renewcommand\normalsize{\fontsize{14}{16.8pt}\selectfont} % Сделать размер шрифта 14

\usepackage{caption} % подписи к рисункам в русской типографской традиции
\DeclareCaptionFormat{GOSTtable}{#2#1\\#3\vspace*{-\baselineskip}}
\DeclareCaptionLabelSeparator{fill}{\hfill}
\DeclareCaptionLabelFormat{fullparents}{\bothIfFirst{#1}{~}#2}

% Начинать каждую главу с новой страницы
%\usepackage{titlesec}
%\newcommand{\sectionbreak}{\clearpage}

\usepackage{xcolor}
\usepackage{suffix}

\newcommand{\setsection}[1]{
	\cleardoublepage
	\phantomsection
	\addcontentsline{toc}{section}{#1}
}

\newcommand{\setsubsection}[1]{
	%\newpage
	\phantomsection
	\addcontentsline{toc}{subsection}{#1}
}

\begin{document}
\setcounter{tocdepth}{3}





%Титульный лист
{
\thispagestyle{empty}

\begin{center}
	
	Министерство образования и науки Российской Федерации\\[0.3cm]
	Государственное образовательное учреждение\\
	высшего профессионального образования\\
	"<Ярославский государственный университет им. П.Г. Демидова">\\
	(ЯрГУ)\\[0.3cm]
	
	Профбюро факультета ИВТ
	
	\bigskip
	\bigskip	
	
	\hspace{15em}"<Принять к исполнению">
	
	\begin{flushright}
		Профорг факультета ИВТ\par
		магистрант 1-го года обучения\par
		\underline{\hspace{3.2cm}}Завойкин Алексей\par
		"<\underline{\hspace{0.5cm}}">\underline{\hspace{3.4cm}}2016 г.\par
	\end{flushright}
	
	\bigskip
	
	{\textbf
		{\textit
			{Проект сценария посвящения первокурсников}
		}
	}
	\\
	
	\bigskip
	\bigskip
	
	{\bf
		Сценарий культурно--массового мероприятия\\<<Посвящение первокурсников 2016>> 
	}
\end{center}

\medskip

\begin{flushright}
	Руководитель разработки сценария\par
	студент 4 курса\par
	\underline{\hspace{3.5cm}}Пащенко Николай\par
	"<\underline{\hspace{0.8cm}}">\underline{\hspace{3.5cm}}2016 г.\par
\end{flushright}

\bigskip 
\bigskip

\begin{flushright}
	Профбюро факультета ИВТ\par
	"<\underline{\hspace{0.8cm}}">\underline{\hspace{3.5cm}}2016 г.\par
\end{flushright}

\vspace{\fill}

\begin{center}
	Ярославль 2016
\end{center}

\begin{flushright}
	\copyright \hspace{0.1cm} Фролов Даниил
\end{flushright}

\clearpage
}





%Содержание
\tableofcontents





%********************************************************************************
\setsection{Основные понятия}
\section*{Основные понятия}

\begin{itemize}
\item \textbf{Панем} --- страна на территории нынешних Соединенных Штатов Америки. Именно в Панеме разворачиваются события романа <<Голодные игры>>\footnote{см. раздел <<Основная сюжетная линия>>}. В нашем случае Панем заменяется на факультет ИВТ.

\item \textbf{Капитолий} --- столица Панема, населенная избалованными жителями, излюбленным развлечением которых является просмотр жестоких сражений между подростками.столица страны. В роли Капитолия выступает лагерь организаторов, сами же организаторы представляют население города (фактически зрителей <<Голодных игр>>).

\item \textbf{Дистрикт} --- один из 12(13) районов Панема, находящийся под контролем Капитолия. Специализируется на конкретной области промышленности.

\item \textbf{Трибут} --- участник игр от конкретного дистрикта.

\item \textbf{Ментор} --- победитель одной из прошлых игр. Наставник трибутов. В нашем случае менторы -- это кураторы групп.
\end{itemize}
%********************************************************************************





%********************************************************************************
\setsection{Отдельные моменты концепции}
\section*{Отдельные моменты концепции}

\begin{itemize}
\item Факультету ИВТ в этом году (2016) исполняется 30 лет. В связи с этим мы говорим, что проводятся \textbf{30-е, юбилейные <<Голодные игры>>}.

\item Так как игры юбилейные, то и правила отличаются от оригинальных. От каждого дистрикта приезжает на 2 участника, а целая команда.

\item Организаторы, которые не являются кураторами (менторами), считаются жителями Капитолия. Необходимо соответствовать этому статусу (в разумных пределах).

\item Профорг и его заместитель, как лица наиболее ответственные, могут не отыгрывать роли жителей Капитолия, так как их первоочередная цель --- обеспечение соблюдения правил проведения мероприятия.

\item Предполагается создание деталей костюма <<жителей Капитолия>>, для большей схожести с прототипами. Достаточно вызывающие, яркие, необычные детали одежды и/или макияжа. \textbf{Данный пункт требует отдельного обсуждения}.

\item В процессе всего мероприятия группы набирают баллы (или иные эквивалентные им единицы). Баллы начисляются за лучший лагерь, лучшую визитку\footnote{см. раздел <<Визитки команд>>}, за успешное выполнение этапа дневной части\footnote{см. раздел <<Дневные этапы>>}, за успехи в спортивных мероприятиях\footnote{см. раздел <<Спортивные мероприятия>>}, за успешное прохождение этапов ночной игры\footnote{см. раздел <<Ночная игра>>}. Возможно добавление иных пунктов получания баллов.

\item Было бы замечательно донести до первокурсников тематику посвята до начала посвята.
\end{itemize}
%********************************************************************************




%********************************************************************************
\setsection{Организаторы}
\section*{Организаторы}



%------------------------------
\setsubsection{Кураторы}
\subsection*{Кураторы}

\begin{itemize}
\item \textbf{ПМИ--11БО}
	\begin{enumerate}
	\item 
	\item 
	\item 
	\end{enumerate}
\item \textbf{ПМИ--12БО}
	\begin{enumerate}
	\item 
	\item 
	\item 
	\end{enumerate}
\item \textbf{ФИИТ--11БО}
	\begin{enumerate}
	\item 
	\item 
	\item 
	\end{enumerate}
\item \textbf{ФИИТ--12БО}
	\begin{enumerate}
	\item 
	\item 
	\item 
	\end{enumerate}
\item \textbf{ПИЭ--11БО}
	\begin{enumerate}
	\item 
	\item 
	\item 
	\end{enumerate}
\end{itemize}
%------------------------------



%------------------------------
\newpage
\setsubsection{\dots и другие}
\subsection*{\dots и другие}

\begin{enumerate}
\item \dots
\item \dots
\item \dots
\end{enumerate}
%------------------------------



%********************************************************************************





%********************************************************************************
\setsection{Основная сюжетная линия}
\section*{Основная сюжетная линия}



%------------------------------
\setsubsection{Введение}
\subsection*{Введение}

\par В качестве сюжета посвята была выбрана тематика романа Сьюзен Коллинз <<Голодный игры>>, первая часть трилогии, а также финал романа.

\par Организаторы (профбюро) выступает в роли Капитолия и его жителей, группы первокурсников в роли трибутов из дистриктов.

\par Каждый из 12(13) фигурирующих в произведении дистриктов (райнов), специализируется на конкретеной отрасли промышеленности:

\begin{enumerate}
\item предметы роскоши для Капитолия;
\item изначально каменщики, затем база подготовки миротворцев;
\item производство электроники, автомобилей, оружия;
\item рыболовство;
\item энергетика;
\item производство транстпортных средств (мало информации в целом);
\item пиломатериалы, древесина, бумага;
\item текстиль, униформа для военных;
\item сельское хозяйство;
\item животноводство;
\item сельское хозяйство (в отличии от 9, возделываются различные культуры, в том числе хлопок);
\item угольная промышленность;
\item официально графит, реально ядерное оружие.
\end{enumerate}

\par Лично я считаю, что дистрикт 6 не стоит выбирать, так как информации о нем недостаточно, кроме того он полностью может быть перекрыт 3. По схожим соображениям дистрикт 9 стоит сразу же исключить из списка, так как он в полной мере может быть заменен 11. Использование дистрикта 13 стоит вынести не обсуждение из-за его обособленности в сюжете романа и привилигированного положения в целом. Также стоит уточнить, что данный дистрикт не принимал участия в <<Голодных играх>>. Не исключаю возможности использования этого дистрикта и в сюжете посвята, в качестве некой организации в стане Капитолия, которая способствовала предполагаемой революции. Итого на данный момент считаю, что в наличии есть 10 дистриктов, из которых необходимо выбрать как \textbf{минимум 5} наиболее простых для реализации первокурсниками и одновременно как можно более интересных с точки зрения отыграша и сюжета в целом.

\par Начиная с 1 сентрября 2016 года начинается подготовка первокурсников к данному мероприятию. Учитывая вышесказанное, организаторам необходимо заранее определить список наиболее подходящих дистриктов. Именно из этого списка первокурсники случайным образом выберут себе тематику группы. Жеребьеву имеет смысл проводить 2 сентября, так как это пятница, дать людям хотя бы одни полные выходные.

\par К моменту проведение жеребьевки первокурсники было бы неплохо кратко, не вдаваясь в подробности, ознакомить первокурсников с тематикой посвята. Закрепление дистрикта за группой планируется следующим образом. Первокурсникам заранее объявляется весь набор возможных дистриктов. В <<шляпу>> кладутся листочки с номерами и кратким описанием отрасли промышленности. Затем группа выдвигает своего представителя, чтобы он вытянул их судьбу. Если группу устраивает выпавшее значение и тематика, то выбор для данной группы окончен, они могут быть свободны. В случае, если выбрать что-либо иной, то представитель группы кладет листок обратно в <<шляпу>> и отправляется в конец очереди. Изначально очерерь формируется следующим образом ПМИ~--~11, ПМИ~--~12, ФИИТ~--~11, ФИИТ~--~12, ПИЭ~--~11. Если я правильно помню, то порядок в очереди никак не влияет на шансы получить тот или иной дистрикт, поэтому очередность чисто формальная.

\par После завершения процедуры выбора тематик группы могут приступать к подготовке к посвяту. Название команды, девиз, внешний вид и визитка команды должны соответствовать выбранной тематике, в разумных пределах, конечно. Кураторам необходимо следить и направлять группы с учетом этого пожелания, насколько это возможно в принципе.
%------------------------------



%------------------------------
\setsubsection{Концепция трех ролей}
\subsection*{Концепция трех ролей}

\par Среди организаторов выделяются 3 роли, исполнители которые более других должны соответствовать своим ролям\footnote{см. раздел <<Роли>>}:
\begin{enumerate}
\item Президент Сноу;
\item Главный распорядитель игр;
\item Цезарь Фликерман (журналист, репортер).
\end{enumerate}
%------------------------------



%------------------------------
\setsubsection{Сюжет}
\subsection*{Сюжет}

\par Около 11-00 первокурсники централизовано автобусами доставляются к месту проведения мероприятия\footnote{см. раздел <<Расписание>>}. После вступительного слова профорга и иных организационных мероприятий организаторам желательно начать отыгрывать роль жителей Капитолия (в разумных пределах).

\par После вступительного слова профорга, а также установки лагерей первокурсниками следует краткая речь президента\footnote{речь президента и распорядителя смотри в разделе <<Роли>>}. Затем речь распорядителя игр.

\par После завершения <<официальной части>> следуют визитки участвников. После каждой визитки репортер (Цезарь Фликерман) задает группе, которая еще не ушла со <<сцены>> несколько (около 3) вопросов\footnote{вопросы для репортера смотри в разделе <<Роли>>}.

\par Затем распорядитель игр объявляет о начале тренировок перед играми, куда входят дневные этапы\footnote{см. раздел <<Дневные этапы>>}, они же <<Вертушка>>.

\par Каждую команду первокурсников ведет по этапам <<Вертушки>> один из их кураторов (менторов). Маршрутные листы будут выданы отдельно\footnote{полные правила изложены в разделе <<Дневные этапы>>}.

\par После <<Вертушки>> следует обед, затем турка, футбол и квидич. Обоснование и мотивацию проведения данных мероприятий участникам объясняет главный распорядитель игр.

\par Затем ужин и непосредственно <<Голодные игры>> в лице Ночной игры.

\par События ночной игры выводят сюжет на революцию, а также казнь президента, возможно распорядителя и репортера. Как и каким образом на данный момент не решено.

\par После революции и казни следует клятва первокурсника, зачитываемая профоргом, после чего завершается отыгрыш ролей в рамках заданной концепции.

\par Команде, набравшей наибольшее количество очков по результатам всего мероприятия выдается приз. В качестве приза предлагается 4 бутылки шампанского или право казнить Сноу (а также распорядители и Цезаря, если будет принято соответствующее решение). Команда может и отказаться от роли палачей. Этот момент не стоит исключать, необходимо отдельно его продумать, хоть я и считаю его маловероятным. Кто же откажется казнить Сноу?
%------------------------------



%********************************************************************************





%********************************************************************************
\setsection{Расписание}
\section*{Расписание}

\begin{enumerate}
\item \textbf{1 сентября -- 9 сентября} --- подготовка первокурсников в соответствии с выбранной тематиков дистрикта;
\item \textbf{8-00} --- подъем в лагере организаторов;
\item \textbf{9-00} --- выезд за первокурсниками;
\item \textbf{9-00 -- 11-00} --- генеральная проверка всех этапов и техники (для визиток и дневных этапов);
\item \textbf{10-00} --- выезд перевокурсников;
\item \textbf{11-00} --- прибытие первокурсников под <<AC/DC~--~High~way~to~Hell>>, вступительная речь профорга;
\item \textbf{11-00 -- 12-00} --- установка лагеря/костра первокурсниками;
\item \textbf{12-00 -- 13-00} --- вступительная речь президента под <<James~Newton~Howard~--~Anthem~Capitol>>, вступительная речь распорядителя, визитки участников, мини--интервью с Цезарем Фликерманом;
\item \textbf{13-00 -- 15-30} --- <<Вертушка>>, дневные этапы;
\item \textbf{15-30 -- 16-30} --- обед;
\item \textbf{16-30 -- 19-00} --- футбол, квидич, турка;
\item \textbf{19-00 -- 20-00} --- ужин (возможно позже);
\item \textbf{20-00 -- 21-30} --- отдых перед ночной;
\item \textbf{21-30 -- 00-30} --- Ночная игра, <<Голодные игры>>;
\item \textbf{00-10 -- 00-30} --- казни, <<клятва первокурсника>>;
\item \textbf{00-30 -- до утра} --- дискотека 70-х, 80-х и 90-х.
\end{enumerate}
%********************************************************************************





%********************************************************************************
\setsection{Роли}
\section*{Роли}



%------------------------------
\setsubsection{Президент Сноу}
\subsection*{Президент Сноу --- \textcolor{lightgray}{(Фролов Даниил)}}

\begin{itemize}
\item \textbf{Внешний вид.} Темный классический костюм, френч, классический ботинки, белая роза в кармане слева. Цвет волос -- на обсуждение.

\item \textbf{Вступительная речь.} В разработке

\item \textbf{Действия в течении игры.} Объяснение правил, контроль за соблюдением регламента, создание атмосферы

\item \textbf{Завершение карьеры.} В конце игры, перед финальным костром будет казнен.
\end{itemize}
%------------------------------



%------------------------------
\setsubsection{Главный распорядитель игр}
\subsection*{Главный распорядитель игр}

\begin{itemize}
\item \textbf{Внешний вид.} На обсуждение. В зависимости от исполнителя роли.

\item \textbf{Речь перед визитками.} В разработке

\item \textbf{Речь перед <<Вертушкой>>.} В разработке

\item \textbf{Речь перед туркой, футболом, квидичем} В разработке

\item \textbf{Речь перед ночной игрой.} В разработке

\item \textbf{Действия в течении игры.} Объяснение правил, речи перед знаковыми частями сценария (см. речь перед...), контроль за соблюдением регламента, создание атмосферы

\item \textbf{Завершение карьеры.} В конце игры \textbf{возможно}, будет казнен.
\end{itemize}
%------------------------------



%------------------------------
\setsubsection{Цезарь Фликерман}
\subsection*{Цезарь Фликерман}

\begin{itemize}
\item \textbf{Внешний вид.} На обсуждение. В зависимости от исполнителя роли.

\item \textbf{Действия в течении игры.} Интервью участников, жителей Капитолия. Во время визиток задает вопросы группам непосредственно на <<сцене>>. Освещение мероприятия в целом.

\item \textbf{Вопросы группам.} В разаработке

\item \textbf{Завершение карьеры.} В конце игры \textbf{возможно}, будет казнен.
\end{itemize}
%------------------------------



%********************************************************************************





%********************************************************************************
\setsection{Визитки команд}
\section*{Визитки команд}

\par Необходимо указать требуемый инвентарь, заранее определить и расчистить место. Уточнить у команд необходимость музыки. Посленее во многом относиться к кураторам.
%********************************************************************************





%********************************************************************************
\setsection{Дневные этапы --- <<Вертушка>>}
\section*{Дневные этапы --- <<Вертушка>>}

\par \textbf{Правила всех этапов объясняются заранее перед тем, как команда приступила к выполнению}
\par За успешное выполнение этапов команде начисляются баллы. Количество и правила начисления баллов будут позже.
\par Дневные этапы стартуют после речи главного распорядителя.
\par На каждый этап отводится 15 минут с учетом передвижения между точками. Сигналом к смене этапа служит <<выстел пушки>>, в случае технических накладок --- SMS-рассылка кураторам, ведущим группы. Крайний случай --- таймеры у организаторов на этапе.



%------------------------------
\setsubsection{БИП}
\subsection*{БИП}

\par Этап заключается в преодолении некой области на земле. На земле сигнальной лентой создается сетка 6x6. Организатор имеет у себя <<карту>> этой сетки. На сетке есть безопасные участки и участки, на которые нельзя вступать. Существует маршрут по соседним клеткам каждая из которых безопасна из левой нижней в верхнюю правую. Клетки считаются соседними, если они соприкасаются стороной. По диагонали --- не соседи.

\par Задача команды \textbf{без слов, молча} найти это маршрут. Если участник во время <<прощупывания>> маршрута <<подорвался>>, то он возращается к команде и \textbf{другой} участник команды продолжает поиск. В итоге, если маршрут найден все <<живые>> участники проходят по маршруту самостоятельно. <<Подорвавшихся>> переносят на спине <<живые>>.

\par Обо всех правилах заранее сообщается во вступительной речи \textbf{проводящего этап}.

\par \textbf{Речь для объяснения правил.} В разработке
%------------------------------



%------------------------------
\setsubsection{Крокодил}
\subsection*{Крокодил}

\par Команда садится в колонну друг за другом.

\par Задача передать впереди сидящего назад

\par Обо всех правилах заранее сообщается во вступительной речи \textbf{проводящего этап}.

\par \textbf{Речь для объяснения правил.} В разработке
%------------------------------



%------------------------------
\setsubsection{Электросеть}
\subsection*{Электросеть}

\par Между двумя деревьями создается сетка из сигнальной ленты с хаотично расположенными отверстиями.

\par Задача передать за сетки (на другую сторону) как можно больше членов команды. Запрещается касаться деревьев и самой сетки. После прохождения участника через отверствие в сетке, оно становится недоступным для дальнейшего использования. Если команда задевает сетки или дерево при передаче, результат аннулируется, этап начинается сначала. Над сеткой и под сеткой --- два разрешенных отверстия. Имеет смысл сразу сообщить об этом.

\par Обо всех правилах заранее сообщается во вступительной речи \textbf{проводящего этап}.

\par \textbf{Речь для объяснения правил.} В разработке
%------------------------------



%------------------------------
\setsubsection{Змейка}
\subsection*{Змейка}

\par Команда становится в колонну друг за другом. Руки на плечи впереди стоящего. Всем кроме первого закрыть глаза.

\par Задача преодолеть маршрут <<змейкой>>. Не подглыдывая!

\par Обо всех правилах заранее сообщается во вступительной речи \textbf{проводящего этап}.

\par \textbf{Речь для объяснения правил.} В разработке
%------------------------------



%------------------------------
\setsubsection{Канаты}
\subsection*{Канаты}

\par Описание в разработке, ибо не видел конкурс в реальности. Жду повторных объяснений

\par 

\par Обо всех правилах заранее сообщается во вступительной речи \textbf{проводящего этап}.

\par \textbf{Речь для объяснения правил.} В разработке
%------------------------------



%------------------------------
\setsubsection{Песня}
\subsection*{Песня}

\par Описание в разработке, ибо не видел конкурс в реальности. Жду повторных объяснений

\par В прошлом году проводил Коля Пашенко, если я не ошибаюсь

\par Обо всех правилах заранее сообщается во вступительной речи \textbf{проводящего этап}.

\par \textbf{Речь для объяснения правил.} В разработке
%------------------------------



%------------------------------
\setsubsection{(Воз)Душный шар}
\subsection*{(Воз)Душный шар --- \textcolor{lightgray}{(Пащенко Николай)}}

\par Описание в разработке. Конкурс предложен Николаем. Хочется описание составлять с ним.

\par 

\par Обо всех правилах заранее сообщается во вступительной речи \textbf{проводящего этап}.

\par \textbf{Речь для объяснения правил.} В разработке
%------------------------------



%------------------------------
\setsubsection{Сценарий}
\subsection*{Сценарий}

\par Как на игре на местности

\par Описание в разработке. Сценарий, который будут изображать команды в разработке.

\par Мария Аверина проводила этот этап во время игры на местности в 2015.

\par Обо всех правилах заранее сообщается во вступительной речи \textbf{проводящего этап}.

\par \textbf{Речь для объяснения правил.} В разработке
%------------------------------



%------------------------------
\setsubsection{Загадки}
\subsection*{Загадки}

\par Описание в разработке.

\par В прошлом году проводила Даша Петухова

\par Обо всех правилах заранее сообщается во вступительной речи \textbf{проводящего этап}.

\par \textbf{Речь для объяснения правил.} В разработке
%------------------------------



%------------------------------
\setsubsection{Конкурс на меткость}
\subsection*{Конкурс на меткость}

\par Команде предлагается поразить мишень из лука. Каждому по 1-2 выстрела. Берем средний результат команды.

\par Необходимо создание большого количества стрел

\par Обо всех правилах заранее сообщается во вступительной речи \textbf{проводящего этап}.

\par \textbf{Речь для объяснения правил.} В разработке
%------------------------------



%********************************************************************************





%********************************************************************************
\setsection{Спортивные мероприятия}
\section*{Спортивные мероприятия}

\par \textbf{Планируется ли проведение волейбола, капитошек, перетягивания каната?}
\par Спортивные этапы предвосхищаются речью главного распорядителя.



%------------------------------
\setsubsection{Турка}
\subsection*{Турка}

Думаю все в курсе, что это такое. Здесь необходимо указать требуемый инвентарь
%------------------------------



%------------------------------
\setsubsection{Футбол}
\subsection*{Футбол}

Думаю все в курсе, что это такое. Здесь необходимо указать требуемый инвентарь
%------------------------------



%------------------------------
\setsubsection{Квидич}
\subsection*{Квидич}

Думаю все в курсе, что это такое. Здесь необходимо указать требуемый инвентарь.\\
Описание желательно составлять с идейным лидером данного мероприятия.\\
Описание правил тоже
%------------------------------



%********************************************************************************





%********************************************************************************
\setsection{Ночная игра --- <<Голодные игры>>}
\section*{Ночная игра --- <<Голодные игры>>}

\par На каждый этап отводится 15 минут с учетом передвижения между точками. Сигналом к смене этапа служит <<выстел пушки>>, в случае технических накладок --- SMS-рассылка кураторам, ведущим группы. Крайний случай --- таймеры у организаторов на этапе.
\par В разработке
%********************************************************************************





\end{document}