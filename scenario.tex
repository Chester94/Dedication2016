\documentclass[a4paper, 14pt]{extarticle}
\usepackage[english, russian]{babel}
\usepackage[utf8x]{inputenc}
\usepackage{fullpage}
\usepackage{indentfirst} % Первый абзац в разделе тоже с красной строки
%\usepackage{cmap} % для кодировки шрифтов в pdf (чтобы не было крокозябры при копировании из pdf )
\usepackage{graphicx} % для вставки картинок
\graphicspath{{res/}} % Путь к папке с картинками
\sloppy % Включение переноса слов в тексте

\title{Сценарий культурно--массового мероприятия <<Посвящение первокурсников 2016>>}
\author{Культурно--массовая комиссия факультета ИВТ}

% Для красивой вставки исходного программного кода 
\usepackage{listings}
\usepackage{color}

\definecolor{mygreen}{rgb}{0,0.6,0}
\definecolor{mygray}{rgb}{0.5,0.5,0.5}
\definecolor{mymauve}{rgb}{0.58,0,0.82}

% Параметры раскрски исходного кода программы 
\lstset{ %
	language = Java,
	extendedchars=\true, %Чтобы русские буквы в комментариях были
	%inputencoding=cp1251,
	%commentstyle=\itshape,
	%stringstyle=\bf,
  backgroundcolor=\color{white},   % choose the background color
  basicstyle=\footnotesize,        % size of fonts used for the code
  %basicstyle=\ttfamily\fontsize{11pt}{11pt}\selectfont,
  breaklines=true,                 % automatic line breaking only at whitespace
  captionpos=b,                    % sets the caption-position to bottom
  commentstyle=\color{mygreen},    % comment style
  escapeinside={\%*}{*)},          % if you want to add LaTeX within your code
  keywordstyle=\color{blue},       % keyword style
  stringstyle=\color{mymauve},     % string literal style
}

\usepackage{hyperref} % Для добавления ссылок в тесте

% Настройка цветов для ссылок
\hypersetup{
colorlinks = true,
linkcolor = black,
pagecolor = black,
urlcolor = blue, 
citecolor = black
}

% Математика
\usepackage{amssymb} % For use "mathbb" function
\usepackage{amsmath}
\usepackage{amsthm}
\usepackage{mathrsfs}
\newcommand{\La}{\mathscr{L}} % Функция Лагранжа
\newcommand{\ls}{{ℓ}} % Красивая l, чтобы легче было отличить от i, 1 b других палок
\providecommand{\norm}[1]{\lVert#1\rVert} % Норма вектора : ||w||
\newcommand{\dpt}[1]{\left\langle#1\right\rangle} % dot product using brackets
\newcommand{\brackets}[1]{\left(#1\right)} % Обернуть скобками автоматического размера
%\newcommand{\dpts}[2]{#2 \langle#1 #2 \rangle} % dot product using brackets with manual size
\newcommand{\R}{\mathbb{R}} % beautiful R for R^n labels
\newcommand{\il}{i = 1, \ldots, \ls} % writes i = 1, ..., l
\newcommand{\ili}{\quad i = 1, \ldots, \ls} % writes i = 1, ..., l with indent in begin
\newcommand{\sumil}{\sum_{i=1}^{\ls}} % Сумма по i, которая изменяется от 1 до l
\newcommand{\minl}{\min\limits} % min with limits under "min" label
\newcommand{\maxl}{\max\limits} % max with limits under "max" label

% Окружение для теорем, определений и т.д.
\theoremstyle{definition}
\newtheorem{definition}{Определение}
\newtheorem{theorem}{Теорема}
\newtheorem{example}{Пример}

% Поправить стиль отрисовки формул (особенно актуально для сумм https://ru.sharelatex.com/learn/Display_style_in_math_mode)
\everymath{\displaystyle}

%\bibliographystyle{unsrt} % упорядочить список использованной литературы по порядку упоминания их в тексте
\bibliographystyle{utf8gost705u}
%\bibliographystyle{utf8gost71u}

\usepackage[labelfont=bf, labelsep=space]{caption} % Делаем надписи "Рис.1" под рисунками жирными и без двоеточия.
\usepackage[top=20mm, bottom=20mm, left=30mm, right=20mm
, nohead % Убрать расстояние для верхних колонтикулов
%, nofoot % Убрать расстояние для нижних колонтикулов
]
{geometry} % Размер полей у старницы
\setlength{\parindent}{1.25cm} % Размер интервала для абзацев 
\usepackage{setspace}
\onehalfspacing
\renewcommand\normalsize{\fontsize{14}{16.8pt}\selectfont} % Сделать размер шрифта 14

\usepackage{caption} % подписи к рисункам в русской типографской традиции
\DeclareCaptionFormat{GOSTtable}{#2#1\\#3\vspace*{-\baselineskip}}
\DeclareCaptionLabelSeparator{fill}{\hfill}
\DeclareCaptionLabelFormat{fullparents}{\bothIfFirst{#1}{~}#2}
\captionsetup[table]{
     format=GOSTtable,
     %font={footnotesize},
     labelformat=fullparents,
     labelsep=fill,
     labelfont=normal,
     textfont=bf,
     justification=centering,
     singlelinecheck=false
     }

% Начинать каждую главу с новой страницы
\usepackage{titlesec}
\newcommand{\sectionbreak}{\clearpage}

\usepackage{xcolor}

% Позволяет подсчитывать число страниц в документе с помощью комманды \pageref*{LastPage}
\usepackage{lastpage}
%\pretolerance 10000

\begin{document}
\setcounter{tocdepth}{3}

% Позволяет подсчитывать число рисунков, таблиц и глав(section) в документе
\newcounter{totfigures}
\newcounter{tottables}
\newcounter{totsections}
\makeatletter
\AtEndDocument{%
	\addtocounter{totfigures}{\value{figure}}%
	\addtocounter{tottables}{\value{table}}%
	\addtocounter{totsections}{\value{section}}%
	\immediate\write\@mainaux{%
		\string\gdef\string\totfig{\number\value{totfigures}}%
		\string\gdef\string\tottab{\number\value{tottables}}%    
		\string\gdef\string\totsections{\number\value{totsections}}%  
	}%
}
\makeatother


%Титульный лист
{
\thispagestyle{empty}

\begin{center}
	
	Министерство образования и науки Российской Федерации\\[0.3cm]
	Государственное образовательное учреждение\\
	высшего профессионального образования\\
	"<Ярославский государственный университет им. П.Г. Демидова">\\
	(ЯрГУ)\\[0.3cm]
	
	Профбюро факультета ИВТ
	
	\bigskip
	\bigskip	
	
	\hspace{15em}"<Принять к исполнению">
	
	\begin{flushright}
		Профорг факультета ИВТ\par
		магистрант 1-го года обучения\par
		\underline{\hspace{3.2cm}}Завойкин Алексей\par
		"<\underline{\hspace{0.5cm}}">\underline{\hspace{3.4cm}}2016 г.\par
	\end{flushright}
	
	\bigskip
	
	{\textbf
		{\textit
			{Проект сценария посвящения первокурсников}
		}
	}
	\\
	
	\bigskip
	\bigskip
	
	{\bf
		Сценарий культурно--массового мероприятия\\<<Посвящение первокурсников 2016>> 
	}
\end{center}

\medskip

\begin{flushright}
	Руководитель разработки сценария\par
	студент 4 курса\par
	\underline{\hspace{3.5cm}}Пащенко Николай\par
	"<\underline{\hspace{0.8cm}}">\underline{\hspace{3.5cm}}2016 г.\par
\end{flushright}

\bigskip 
\bigskip

\begin{flushright}
	Профбюро факультета ИВТ\par
	"<\underline{\hspace{0.8cm}}">\underline{\hspace{3.5cm}}2016 г.\par
\end{flushright}

\vspace{\fill}

\begin{center}
	Ярославль 2016
\end{center}

\begin{flushright}
	\copyright \hspace{0.1cm} Фролов Даниил
\end{flushright}

\clearpage
}


%Содержание
\tableofcontents



\section{Основные понятия, термины, ключевые моменты}

\textbf{Основные понятия, термины}
\begin{itemize}
\item \textbf{Культурно--массовое мероприятие <<Посвящение первокурсников 2016>>} --- в дальнейшем мероприятие или посвят

\item \textbf{Панем} --- страна, в которой разварачиваются собития романа. В нашей концепции Панем -- факультет ИВТ.\\<<Панем сегодня, Панем завтра, Панем -- всегда!>> превращаем в <<ИВТ сегодня, ИВТ завтра, ИВТ -- всегда!>>

\item \textbf{Капитолий} --- столица страны. В ролик Капитолия выступает лагерь организаторов. организаторы -- жители Капитолия.

\item \textbf{Дистиркты} --- префектура, провинция, зависимая часть. Район страны, находящийся под контролем Капитолия, лишем самостоятельности. Специализируется на конкретной области промышленности

\item \textbf{Трибут} --- участник игр от конкретного дистрикта. Фактически у нас это группа в целом. Например ПМИ--11БО --- некий дистрикт.

\item \textbf{Менторы} --- победители прошлых игр. Наставники трибутов. В наших реалиях -- кураторы. Возможно, имеет смысл на время подготовки использовать <<менторы>>?


\end{itemize}


\textbf{Ключевые моменты}
\begin{itemize}
\item Не забываем про <<ИВТ сегодня, ИВТ завтра, ИВТ -- всегда!>> и <<Счастливых вам Голодных игр, и пусть удача всегда будет на вашей стороне>>

\item Предполагаются некоторые детали костюмов для организаторов для придания их внешности вычурности и вызова, которые присущи жителям Капитолия

\item Организаторам имеет смысл отыгрывать роли жителей Капитолия, насколько это возможно, в разумных пределах

\item Профорг и заместитель могут действовать вне концепции сюжета

\item Мы проводим Юбилейные 30-е Голодные игры (факультету 30 лет)

\item За успешное прохождение этапов (дневных и ночных), за лучший лагерь, лучшую визитку команде начиляются баллы
\end{itemize}


\section{Основная сюжетная линия}

\par В качестве сюжета посвята была выбрана тематика романа Сьюзен Коллинз <<Голодный игры>>, первая часть трилогии, а также финал романа.
\par Организаторы (профбюро) выступает в роли Капитолия, первокурсники в роли дистриктов.
\par Начиная с 1 сентрября 2016 года начинается подготовка первокурсников к данному мероприятию. Каждый из 12(13) фигурирующих в произведении дистриктов (райнов), специализируется на конкретеной отрасли промышеленности:

\begin{enumerate}
\item предметы роскоши для Капитолия
\item изначально каменщики, затем база подготовки миротворцев
\item производство электроники, автомобилей, оружия
\item рыболовство
\item энергетика
\item производство транстпортных средств (мало информации в целом)
\item пиломатериалы, древесина, бумага
\item текстиль, униформа для военных
\item сельское хозяйство
\item животноводство
\item сельское хозяйство (в отличии от 9, возделываются разлычные культуры, в том числе хлопок)
\item угольная промышленность
\item официально графит, реально ядерное оружие
\end{enumerate}

\par Из 12(13) дистриктов организаторам необходимо выбрать наиболее интересные. В течении первых дней учебы группы первокурсников с помощью жребия выберут себе <<тематику>>.

\par Например, организаторами будет выбрано 7 <<интересных>> дистриктов. В таком случае представитель из каждой группы (очередность стоит определить отдельно) тянет <<тематику>>, если группу устраивает выпавшее направление, то для них выбор окончен, если же группа желает выбрать нечто иное, представитель отправляется в конец очереди.

\par По результатам жеребьевки группы начинают подготовку к посвяту. Название, девиз, внешний вид, а также визитка должны соответствовать выбранной тематике. Кураторам необходимо следить и направлять группы с учетом этого требования.

\par Само мероприятие проходит в лесополосе за пределами города. Около 11-00 первокурсники централизовано автобусами доставляются к месту проведения мероприятия\footnote{также смотри раздел <<Расписание>>}. После вступительного слова профорга и иных организационных мероприятий организаторам желательно начать отыгрывать роль жителя Капитолия (в разумных пределах).

\par Среди организаторов выделяются 3 роли, которые более других должны отыгрывать свои роли\footnote{подробнее о ролях в разделе <<Роли>>}:
\begin{enumerate}
\item Президент Сноу,
\item Главный распорядитель игр,
\item Цезарь Фликерман (журналист, репортер).
\end{enumerate}

\par Отдельно напоминаю, что профорг и его заместитель находятся вне концепции сценария. Следят за порядком и т.д., не отвлекаясь на отыгрывание жителей Капитолия (если есть желание отыграть роль жителя Капитолия -- это замечательно).

\par После вступительного слова профорга, а также установки лагерей первокурсниками идет краткая речь президента\footnote{речь президента и распорядителя смотри в разделе <<Роли>>}. Затем речь распорядителя игр.

\par После завершения <<официальной части>> следуют визитки участвников. После каждой визитки репортер на камеру задает группе, которая еще не ушла со <<сцены>> несколько (около 3) вопросов\footnote{вопросы для репортера смотри в разделе <<Роли>>}.

\par Затем распорядитель игр объявляет о начале тренировок перед играми, куда входят дневные этапы\footnote{подробнее о дневных этапах смотри в разделе <<Дневные этапы>>}, они же <<Вертушка>>.

\par После <<Вертушки>> следует обед, затем турка, футбол и квидич. Обоснование и мотивацию проведения данных мероприятий участникам объясняет главный распорядитель игр.

\par Затем ужин и непосредственно <<Голодные игры>> в лице Ночной игры.

\par События ночной игры выводят сюжет на революцию, а также казнь президента, а возможно распорядителя и репортера. Как и каким образом на данный момент не решено.

\par После революции и казни следует клятва первокурсника, зачитываемая профоргом, после чего завершается отыгрыш ролей в рамках заданной концепции.

\par Во время мероприятия команды зарабатывает баллы (или что-нибудь еще, уточняем). По итогам лучшей команде выдается приз (был вариант 4 бутылки шампанского).


\section{Расписание}

\begin{enumerate}
\item \textbf{1 сентября -- 9 сентября} --- подготовка первокурсников в соответствии с выбранной тематиков дистрикта
\item \textbf{8-00} --- подъем в лагере организаторов
\item \textbf{9-00} --- выезд за первокурсниками
\item \textbf{9-00 -- 11-00} --- генеральная проверка всех этапов и техники (для визиток и дневных этапов)
\item \textbf{10-00} --- выезд перевокурсников
\item \textbf{11-00} --- прибытие первокурсников под <<AC/DC -- High way to Hell>>, вступительная речь профорга
\item \textbf{11-00 -- 12-00} --- установка лагеря/костра первокурсниками
\item \textbf{12-00 -- 13-00} --- вступительная речь президента под <<The Hunger Games -- Anthem Capitol>>, вступительная речь распорядителя под <<Arcade Fire -- Abraham's Daughter>>, визитки участников, мини--интервью с Цезарем
\item \textbf{13-00 -- 15-30} --- <<Вертушка>>, дневные этапы
\item \textbf{15-30 -- 16-30} --- обед
\item \textbf{16-30 -- 19-00} --- футбол, квидич, турка
\item \textbf{19-00 -- 20-00} --- ужин (возможно позже)
\item \textbf{20-00 -- 21-30} --- отдых перед ночной
\item \textbf{21-30 -- 00-30} --- Ночная игра, <<Голодные игры>>
\item \textbf{00-10 -- 00-30} --- казни, <<клятва первокурсника>>
\item \textbf{00-30 -- до утра} --- дискотека 70-х, 80-х и 90-х.
\end{enumerate}


\section{Роли}

\subsection{Президент Сноу}
\textcolor{lightgray}{(Фролов Даниил)}

\par \textbf{Внешний вид.} Темный классический костюм, френч, классический ботинки, белая роза в кармане слева. Цвет волос -- на обсуждение.

\par \textbf{Вступительная речь.} В разработке

\par \textbf{Действия в течении игры.} Объяснение правил, контроль за соблюдением регламента, создание атмосферы

\par \textbf{Завершение карьеры.} В конце игры, перед финальным костром будет казнен.


\subsection{Главный распорядитель игр}

\par \textbf{Внешний вид.} На обсуждение. В зависимости от исполнителя роли.

\par \textbf{Речь перед визитками.} В разработке

\par \textbf{Речь перед <<Вертушкой>>.} В разработке

\par \textbf{Речь перед туркой, футболом, квидичем} В разработке

\par \textbf{Речь перед ночной игрой.} В разработке

\par \textbf{Действия в течении игры.} Объяснение правил, речи перед знаковыми частями сценария (см. речь перед...), контроль за соблюдением регламента, создание атмосферы

\par \textbf{Завершение карьеры.} В конце игры \textbf{возможно}, будет казнен.


\subsection{Цезарь Фликерман}

\par \textbf{Внешний вид.} На обсуждение. В зависимости от исполнителя роли.

\par \textbf{Действия в течении игры.} Интервью участников, жителей Капитолия. Во время визиток задает вопросы группам непосредственно на <<сцене>>. Освещение мероприятия в целом.

\par \textbf{Вопросы группам.} В разаработке

\par \textbf{Завершение карьеры.} В конце игры \textbf{возможно}, будет казнен.



\section{Визитки команд}
\par Необходимо указать требуемый инвентарь, заранее определить и расчистить место. Уточнить у команд необходимость музыки. Посленее во многом относиться к кураторам.




\section{Дневные этапы --- <<Вертушка>>}

\par \textbf{Правила всех этапов объясняются заранее перед тем, как команда приступила к выполнению}
\par За успешное выполнение этапов команде начисляются баллы. Количество и правила начисления баллов будут позже.
\par Дневные этапы стартуют после речи главного распорядителя.
\par На каждый этап отводится 15 минут с учетом передвижения между точками. Сигналом к смене этапа служит <<выстел пушки>>, в случае технических накладок --- SMS-рассылка кураторам, ведущим группы. Крайний случай --- таймеры у организаторов на этапе.

\subsection{БИП}

\par Этап заключается в преодолении некой области на земле. На земле сигнальной лентой создается сетка 6x6. Организатор имеет у себя <<карту>> этой сетки. На сетке есть безопасные участки и участки, на которые нельзя вступать. Существует маршрут по соседним клеткам каждая из которых безопасна из левой нижней в верхнюю правую. Клетки считаются соседними, если они соприкасаются стороной. По диагонали --- не соседи.

\par Задача команды \textbf{без слов, молча} найти это маршрут. Если участник во время <<прощупывания>> маршрута <<подорвался>>, то он возращается к команде и \textbf{другой} участник команды продолжает поиск. В итоге, если маршрут найдет все <<живые>> участники проходят по маршруту самостоятельно. <<Подорвавшихся>> переносят на спине <<живые>>.

\par Обо всех правилах заранее сообщается во вступительной речи \textbf{проводящего этап}.

\par \textbf{Речь для объяснения правил.} В разработке


\subsection{Крокодил}

\par Команда садится в колонну друг за другом.

\par Задача передать впереди сидящего назад

\par Обо всех правилах заранее сообщается во вступительной речи \textbf{проводящего этап}.

\par \textbf{Речь для объяснения правил.} В разработке


\subsection{Электросеть}

\par Между двумя деревьями создается сетка из сигнальной ленты с хаотично расположенными отверстиями.

\par Задача передать за сетки (на другую сторону) как можно больше членов команды. Запрещается касаться деревьев и самой сетки. После прохождения участника через отверствие в сетке, оно становится недоступным для дальнейшего использования. Если команда задевает сетки или дерево при передаче, результат аннулируется, этап начинается сначала. Над сеткой и под сеткой --- два разрешенных отверстия. Имеет смысл сразу сообщить об этом.

\par Обо всех правилах заранее сообщается во вступительной речи \textbf{проводящего этап}.

\par \textbf{Речь для объяснения правил.} В разработке


\subsection{Змейка}

\par Команда становится в колонну друг за другом. Руки на плечи впереди стоящего. Всем кроме первого закрыть глаза.

\par Задача преодолеть маршрут <<змейкой>>. Не подглыдывая!

\par Обо всех правилах заранее сообщается во вступительной речи \textbf{проводящего этап}.

\par \textbf{Речь для объяснения правил.} В разработке



\subsection{Канаты}

\par Описание в разработке, ибо не видел конкурс в реальности. Жду повторных объяснений

\par 

\par Обо всех правилах заранее сообщается во вступительной речи \textbf{проводящего этап}.

\par \textbf{Речь для объяснения правил.} В разработке



\subsection{Песня}

\par Описание в разработке, ибо не видел конкурс в реальности. Жду повторных объяснений

\par В прошлом году проводил Коля Пашенко, если я не ошибаюсь

\par Обо всех правилах заранее сообщается во вступительной речи \textbf{проводящего этап}.

\par \textbf{Речь для объяснения правил.} В разработке




\subsection{(Воз)Душный шар}
\textcolor{lightgray}{(Пащенко Николай)}
\par Описание в разработке. Конкурс предложен Николаем. Хочется описание составлять с ним.

\par 

\par Обо всех правилах заранее сообщается во вступительной речи \textbf{проводящего этап}.

\par \textbf{Речь для объяснения правил.} В разработке



\subsection{Сценарий <<как на игре на местности>>}

\par Описание в разработке. Сценарий, который будут изображать команды в разработке.

\par Мария Аверина проводила этот этап во время игры на местности в 2015.

\par Обо всех правилах заранее сообщается во вступительной речи \textbf{проводящего этап}.

\par \textbf{Речь для объяснения правил.} В разработке



\subsection{Загадки}

\par Описание в разработке.

\par В прошлом году проводила Даша Петухова

\par Обо всех правилах заранее сообщается во вступительной речи \textbf{проводящего этап}.

\par \textbf{Речь для объяснения правил.} В разработке



\subsection{Конкурс на меткость}

\par Команде предлагается поразить мишень из лука. Каждому по 1-2 выстрела. Берем средний результат команды.

\par Необходимо создание большого количества стрел

\par Обо всех правилах заранее сообщается во вступительной речи \textbf{проводящего этап}.

\par \textbf{Речь для объяснения правил.} В разработке



\section{Спортивные мероприятия}
\par \textbf{Планируется ли проведение волейбола, капитошек, перетягивания каната?}
\par Спортивные этапы предвосхищаются речью главного распорядителя.

\subsection{Турка}
Думаю все в курсе, что это такое. Здесь необходимо указать требуемый инвентарь

\subsection{Футбол}
Думаю все в курсе, что это такое. Здесь необходимо указать требуемый инвентарь

\subsection{Квидич}
Думаю все в курсе, что это такое. Здесь необходимо указать требуемый инвентарь.\\
Описание желательно составлять с идейным лидером данного мероприятия.\\
Описание правил тоже



\section{Ночная игра --- <<Голодные игры>>}
\par На каждый этап отводится 15 минут с учетом передвижения между точками. Сигналом к смене этапа служит <<выстел пушки>>, в случае технических накладок --- SMS-рассылка кураторам, ведущим группы. Крайний случай --- таймеры у организаторов на этапе.
\par В разработке


\end{document}