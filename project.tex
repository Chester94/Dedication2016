\documentclass[a4paper, 12pt]{extarticle}
\usepackage{preamble}

\begin{document}
\setcounter{tocdepth}{3}





%Титульный лист
{
\thispagestyle{empty}

\begin{center}
	
	Министерство образования и науки Российской Федерации\\[0.3cm]
	Государственное образовательное учреждение\\
	высшего профессионального образования\\
	<<Ярославский государственный университет им. П.Г. Демидова>>\\
	(ЯрГУ)\\[0.3cm]
	
	Профбюро факультета ИВТ
	
	\bigskip
	\bigskip	
	
	\begin{flushright}
		<<Принять к исполнению>>
	\end{flushright}
	
	\begin{flushright}
		Профорг факультета ИВТ\par
		магистрант 1-го года обучения\par
		\underline{\hspace{3.2cm}}Завойкин Алексей\par
		<<\underline{\hspace{0.5cm}}>>\underline{\hspace{3.4cm}}2016 г.\par
	\end{flushright}
	
	\bigskip
	
	{\textbf
		{\textit
			{Проект культурно--массового мероприятия}
		}
	}
	\\
	
	\bigskip
	\bigskip
	
	{\bf
		<<Посвящение первокурсников 2016>> 
	}
\end{center}

\medskip

\begin{flushright}
	Руководитель разработки сценария\par
	студент 4 курса\par
	\underline{\hspace{3.5cm}}Пащенко Николай\par
	<<\underline{\hspace{0.8cm}}>>\underline{\hspace{3.5cm}}2016 г.\par
\end{flushright}

\bigskip 
\bigskip

\begin{flushright}
	Профбюро факультета ИВТ\par
	<<\underline{\hspace{0.8cm}}>>\underline{\hspace{3.5cm}}2016 г.\par
\end{flushright}

\vspace{\fill}

\begin{center}
	Ярославль 2016
\end{center}

\begin{flushright}
	\copyright \hspace{0.1cm} Фролов Даниил
\end{flushright}

\clearpage
}





%Содержание
\tableofcontents

%Организаторы
\newsection{Организаторы}{Организаторы}

\par \textbf{Профорг} --- Завойкин Алексей, 8-910-963-56-80\\
\par \textbf{Заместитель профорга} --- Второй Сергей, 8-920-128-76-04



%------------------------------
\newsubsection{Кураторы}{Кураторы}

\begin{itemize}
\item \textbf{ПМИ--11БО}
	\begin{enumerate}
	\item 
	\item 
	\item 
	\end{enumerate}
\item \textbf{ПМИ--12БО}
	\begin{enumerate}
	\item 
	\item 
	\item 
	\end{enumerate}
\item \textbf{ФИИТ--11БО}
	\begin{enumerate}
	\item 
	\item 
	\item 
	\end{enumerate}
\item \textbf{ФИИТ--12БО}
	\begin{enumerate}
	\item 
	\item 
	\item 
	\end{enumerate}
\item \textbf{ПИЭ--11БО}
	\begin{enumerate}
	\item 
	\item 
	\item 
	\end{enumerate}
\end{itemize}
%------------------------------



%------------------------------
\newpage
\newsubsection{\dots и все вместе}{\dots и все вместе}

\begin{enumerate}
\item Белова (Морозова) Мария, 8-906-631-84-10
\item Быкадорова Анастасия, 8-910-968-15-48
\item Второй Сергей, 8-920-128-76-04
\item Гончаров Александр, 8-915-997-56-26
\item Грабовецкая Вера, 8-996-926-76-57
\item Григорова Наталья, 8-930-119-67-38, 8-910-826-15-08
\item Дроздов Семён, 8-961-160-61-30
\item Завойкин Алексей, 8-910-963-56-80
\item Иванов Константин, 8-903-691-08-45
\item Кабанов Георгий, 8-930-107-63-97
\item Краснов Игнат, 8-915-962-02-66
\item Ковалёва (Миллер) Алла, 8-909-279-64-50, 8-920-121-76-45
\item Козлов Александр, 8-905-133-57-79
\item Козлов Глеб, 8-930-112-38-62
\item Коротков Александр, 8-980-650-12-73
\item Кудряшова Дарья, 8-961-026-40-00
\item Кузнецова Мария, 8-905-130-30-25
\item Кучма Марина, 8-999-785-81-71
\item Лайминг Влад, 8-920-112-69-65
\item Мосейкова Марина, 8-915-973-47-57
\item Новиков Андрей, 8-915-977-05-57
\item Новикова Наталья, 8-961-162-44-21
\item Пащенко Николай, 8-920-118-23-17
\item Петрова Ксения, 8-999-234-09-38
\item Сажин Сергей, 8-903-692-29-50
\item Смирнова Валерия, 8-910-963-56-11
\item Тепленёв Никита, 8-980-663-76-41
\item Титова Елизавета, 8-905-135-35-22
\item Томчук Анна, 8-920-128-02-03
\item Фролов Даниил, 8-909-276-61-16
\item Худоян Эдгар, 8-930-117-61-10
\item Чехранов Серегей, 8-920-145-69-17
\item Чудаков Виталий, 8-961-160-72-97
\item Шляпников Валерий, 8-905-633-00-23
\item Шурыгина Алена, 8-930-120-68-30
\item Юркина Светлана, 8-915-974-10-54
\end{enumerate}
%------------------------------

%Основные понятия
\newsection{Основные понятия}{Основные понятия}

\begin{itemize}
\item \textbf{Панем} --- страна на территории нынешних Соединенных Штатов Америки. Именно в Панеме разворачиваются события романа <<Голодные игры>>\footnote{см. раздел <<Основная сюжетная линия>>}. В нашем случае Панем заменяется на факультет ИВТ.

\item \textbf{Капитолий} --- столица Панема, населенная избалованными жителями, излюбленным развлечением которых является просмотр жестоких сражений между подростками, столица страны. В роли Капитолия выступает лагерь организаторов, сами же организаторы представляют население города (фактически зрителей <<Голодных игр>>).

\item \textbf{Дистрикт} --- один из 12(13) районов Панема, находящийся под контролем Капитолия. Специализируется на конкретной области промышленности.

\item \textbf{Трибут} --- участник игр от конкретного дистрикта.

\item \textbf{Ментор} --- победитель одной из прошлых игр. Наставник трибутов. В нашем случае менторы -- это кураторы групп.
\end{itemize}

%Отдельные моменты концепции
\newsection{Отдельные моменты концепции}{Отдельные моменты концепции}

\begin{itemize}
\item Факультету ИВТ в этом году (2016) исполняется 30 лет. В связи с этим мы говорим, что проводятся \textbf{30-е, юбилейные <<Голодные игры>>}.

\item Так как игры юбилейные, то и правила отличаются от оригинальных. От каждого дистрикта приезжает не 2 участника, а целая команда.

\item Организаторы, которые не являются кураторами (менторами), считаются жителями Капитолия. Необходимо соответствовать этому статусу (в разумных пределах).

\item Профорг и его заместитель, как лица наиболее ответственные, могут не отыгрывать роли жителей Капитолия, так как их первоочередная цель --- обеспечение соблюдения правил проведения мероприятия.

\item Предполагается создание деталей костюма <<жителей Капитолия>>, для большей схожести с прототипами. Достаточно вызывающие, яркие, необычные детали одежды и/или макияжа.

\item В процессе всего мероприятия группы набирают баллы (или иные эквивалентные им единицы). Баллы начисляются за лучший лагерь, лучшую визитку\footnote{см. раздел <<Визитки команд>>}, за успешное выполнение этапа дневной части\footnote{см. раздел <<Дневные этапы>>}, за успехи в спортивных мероприятиях\footnote{см. раздел <<Спортивные мероприятия>>}, за успешное прохождение этапов ночной игры\footnote{см. раздел <<Ночная игра>>}. Возможно добавление иных пунктов получения баллов.

\item Было бы замечательно донести до первокурсников тематику посвята до начала посвята.

\item <<ИВТ сегодня, ИВТ завтра, ИВТ всегда>>.

\item <<Счастливых вам <<Голодных игр>>, и пусть удача всегда будет на вашей стороне>>.
\end{itemize}

%Основная сюжетная линия
\newsection{Основная сюжетная линия}{Основная сюжетная линия}



%********************************************************************************
\newsubsection{Введение}{Введение}

\par В качестве сюжета посвята была выбрана тематика романа Сьюзен Коллинз <<Голодные игры>>: первая часть трилогии, а также финал романа.

\par Организаторы (профбюро) выступают в роли Капитолия и его жителей, группы первокурсников --- в роли трибутов из дистриктов.

\par Каждый из 12(13) фигурирующих в произведении дистриктов (районов), специализируется на конкретной отрасли промышленности. Из имеющихся выбрано 10 дистриктов:

\begin{itemize}
\item \textbf{Дистрикт №1} предметы роскоши для Капитолия;
\item \textbf{Дистрикт №2} изначально каменщики, затем база подготовки миротворцев;
\item \textbf{Дистрикт №3} производство электроники, автомобилей, оружия;
\item \textbf{Дистрикт №4} рыболовство;
\item \textbf{Дистрикт №5} энергетика;
\item \textbf{Дистрикт №7} пиломатериалы, древесина, бумага;
\item \textbf{Дистрикт №8} текстиль, униформа для военных;
\item \textbf{Дистрикт №9} животноводство;
\item \textbf{Дистрикт №11} сельское хозяйство;
\item \textbf{Дистрикт №12} угольная промышленность;
\end{itemize}

\par Начиная с 1 сентрября 2016 года, осуществляется подготовка первокурсников к данному мероприятию. Учитывая вышесказанное, организаторам необходимо заранее определить список наиболее подходящих дистриктов. Именно из этого списка первокурсники случайным образом выберут себе тематику группы. Жеребьевку имеет смысл проводить 2 сентября, так как это пятница, что позволит дать людям хотя бы одни полные выходные.

\par К моменту проведения жеребьевки было бы неплохо кратко, не вдаваясь в подробности, ознакомить первокурсников с тематикой посвята. Закрепление дистрикта за группой планируется следующим образом. Первокурсникам заранее объявляется весь набор возможных дистриктов. В <<шляпу>> кладутся листочки с номерами и кратким описанием отрасли промышленности. Затем группа выдвигает своего представителя, чтобы он вытянул их судьбу. Если группу устраивает выпавшее значение и тематика, то выбор для данной группы окончен, они могут быть свободны. В случае, если группа не согласна с выбором, то представитель группы кладет листок обратно в <<шляпу>> и отправляется в конец очереди. Изначально очередь формируется следующим образом ПМИ~--~11, ПМИ~--~12, ФИИТ~--~11, ФИИТ~--~12, ПИЭ~--~11. Если я правильно помню, то порядок в очереди никак не влияет на шансы получить тот или иной дистрикт, поэтому очередность чисто формальная.

\par После завершения процедуры выбора тематик группы могут приступать к подготовке к посвяту. Название команды, девиз, внешний вид и визитка команды должны соответствовать выбранной тематике, в разумных пределах, конечно. Кураторам необходимо следить и направлять группы с учетом этого пожелания, насколько это возможно в принципе.
%********************************************************************************



%********************************************************************************
\newsubsection{Концепция трех ролей}{Концепция трех ролей}

\par Среди организаторов выделяются 3 роли, исполнители которые более других должны соответствовать своим ролям\footnote{см. раздел <<Роли>>}:
\begin{enumerate}
\item Президент Сноу;
\item Главный распорядитель игр;
\item Цезарь Фликерман (журналист, репортер).
\end{enumerate}

\par Кроме того планируется четвертая эпизодическая роль --- Предводитель повстанцев (революционеров). Именно он скажет, что во время <<Голодных игр>>, пока вся страна наблюдала за кровавой резней, группа из бывших победителей -- менторов, кураторов -- свергла власть и захватила в плен Сноу и его приближенных.
%********************************************************************************



%********************************************************************************
\newsubsection{Краткое содержание сюжета}{Краткое содержание сюжета}

\par Около 11-00 первокурсники централизовано автобусами доставляются к месту проведения мероприятия\footnote{см. раздел <<Расписание>>}. После вступительного слова профорга и иных организационных мероприятий организаторам желательно начать отыгрывать роль жителей Капитолия (в разумных пределах).

\par После установки первокурсниками лагерей следует краткая речь президента\footnote{речь президента и распорядителя смотри в разделе <<Роли>>}. Затем --- речь распорядителя игр.

\par После завершения <<официальной части>> следуют визитки участников. После каждой визитки репортер (Цезарь Фликерман) задает группе, которая еще не ушла со <<сцены>>, несколько (около 3) вопросов\footnote{вопросы для репортера смотри в разделе <<Роли>>}. Фактически Цезарь полностью проводит визитки: объявление, представление команд, общий конферанс.

\par Затем распорядитель игр объявляет о начале тренировок перед играми, куда входят дневные этапы\footnote{см. раздел <<Дневные этапы>>}, они же --- <<Вертушка>>.

\par Каждую команду первокурсников ведет по этапам <<Вертушки>> один из их кураторов (менторов). Маршрутные листы будут выданы отдельно\footnote{полные правила изложены в разделе <<Дневные этапы>>}.

\par После <<Вертушки>> следует обед, затем турка, футбол и квиддич. Обоснование и мотивацию проведения данных мероприятий участникам объясняет главный распорядитель игр.

\par Затем ужин и непосредственно <<Голодные игры>> в виде Ночной игры.

\par События ночной игры выводят сюжет на революцию. После прибытия всех первокурсников на место финального костра следует финал сюжета. Лидер повстанцев, захвативших власть и свергнувших президента, выступает перед группами и рассказывает о событиях, <<происходивших>> параллельно с <<Голодными играми>>. Он же объявляет и финал сюжета.

\par В качестве финала предлагается публичная казнь Кориолануса Сноу через повешение. Победившая команда получает право решить судьбу президента. Они могут его помиловать. Главный распорядитель и Цезарь Фликерман вообще никак не <<наказываются>>.
%********************************************************************************

%Расписание
\newsection{Расписание}{Расписание}

\begin{enumerate}
\item \textbf{2 сентября -- 9 сентября} --- подготовка первокурсников в соответствии с выбранной тематикой дистрикта;
\item \textbf{8-00} --- подъем в лагере организаторов;
\item \textbf{9-00} --- выезд за первокурсниками;
\item \textbf{9-00 -- 11-00} --- генеральная проверка всех этапов и техники (для визиток и дневных этапов);
\item \textbf{10-00} --- выезд первокурсников;
\item \textbf{11-00} --- прибытие первокурсников под <<AC/DC~--~High~way~to~Hell>>, вступительная речь профорга;
\item \textbf{11-00 -- 12-00} --- установка лагеря/костра первокурсниками;
\item \textbf{12-00 -- 13-00} --- вступительная речь президента под <<James~Newton~Howard~--~Anthem~Capitol>>, вступительная речь распорядителя, визитки участников, мини--интервью с Цезарем Фликерманом;
\item \textbf{13-00 -- 15-30} --- <<Вертушка>>, дневные этапы;
\item \textbf{15-30 -- 16-30} --- обед;
\item \textbf{16-30 -- 19-00} --- футбол, квиддич, турка;
\item \textbf{19-00 -- 20-00} --- ужин (возможно, позже);
\item \textbf{20-00 -- 21-30} --- отдых перед ночной игрой;
\item \textbf{21-30 -- 00-30} --- Ночная игра <<Голодные игры>>;
\item \textbf{00-10 -- 00-30} --- финал сюжета, <<клятва первокурсника>>;
\item \textbf{00-30 -- до утра} --- дискотека.
\end{enumerate}





%********************************************************************************
\newsection{Роли}{Роли}



%------------------------------
\newsubsection{Президент Сноу --- (Фролов Даниил, 8-909-276-61-16)}{Президент Сноу}

\begin{itemize}
\item \textbf{Внешний вид.} Темный классический костюм, френч, классические ботинки, белая роза в кармане слева. Цвет волос -- на обсуждение.

\item \textbf{Вступительная речь.}\\
Два момента речи\\
Мы салютуем вашей отваге и вашей жертвенности (про поступление на наш факультет).\\
ИВТ сегодня, ИВТ завтра, ИВТ всегда!

\item \textbf{Действия в течение игры.} Объяснение правил, контроль за соблюдением регламента, создание атмосферы

\item \textbf{Завершение карьеры.} В зависимости от выбранного финала.
\end{itemize}
%------------------------------



%------------------------------
\newsubsection{Главный распорядитель игр}{Главный распорядитель игр}

\begin{itemize}
\item \textbf{Внешний вид.} На обсуждение. В зависимости от исполнителя роли.

\item \textbf{Речь перед визитками.} В разработке

\item \textbf{Речь перед <<Вертушкой>>.} В разработке

\item \textbf{Речь перед туркой, футболом, квиддичем} В разработке

\item \textbf{Речь перед Ночной игрой.} В разработке

\item \textbf{Действия в течение игры.} Объяснение правил, речи перед знаковыми частями сценария (см. речь перед...), контроль за соблюдением регламента, создание атмосферы

\item \textbf{Завершение карьеры.} В зависимости от выбранного финала.
\end{itemize}
%------------------------------



%------------------------------
\newsubsection{Цезарь Фликерман}{Цезарь Фликерман}

\begin{itemize}
\item \textbf{Внешний вид.} На обсуждение. В зависимости от исполнителя роли.

\item \textbf{Действия в течение игры.} Интервью участников, жителей Капитолия. Проведение визиток, интервьюирование  групп, находящихся непосредственно на <<сцене>>. Освещение мероприятия в целом.

\item \textbf{Вопросы группам.} В разработке

\item \textbf{Завершение карьеры.} В зависимости от выбранного финала.
\end{itemize}
%------------------------------



%------------------------------
\newsubsection{Лидер повстанцев --- (Завойкин Алексей)}{Лидер повстанцев}

\begin{itemize}
\item \textbf{Внешний вид.} Вероятно, не регламентируется.

\item \textbf{Действия в течение игры.} Вероятно, исполнитель --- профорг. Контроль за соблюдением регламента без отыгрывания роли.

\item \textbf{Финальная речь.} В зависимости от выбранного финала. В разработке.
\end{itemize}
%------------------------------



%********************************************************************************





%********************************************************************************
\newsection{Визитки команд}{Визитки команд}

\par Необходимо указать требуемый инвентарь, заранее определить и расчистить место. Уточнить у команд необходимость музыки. Последнее во многом относиться к кураторам.
%********************************************************************************





%********************************************************************************
\newsection{Дневные этапы --- <<Вертушка>>}{Дневные этапы --- <<Вертушка>>}

\par \textbf{Правила всех этапов объясняются заранее, перед тем, как команда приступила к выполнению.}
\par За успешное выполнение этапов команде начисляются баллы. Количество и правила начисления баллов будут позже.
\par Дневные этапы стартуют после речи главного распорядителя.
\par На каждый этап отводится 15 минут с учетом передвижения между точками. Сигналом к смене этапа служит <<выстрел пушки>>, в случае технических накладок --- SMS-рассылка кураторам, ведущим группы. Крайний случай --- таймеры у организаторов на этапе.



%------------------------------
\newsubsection{БИП}{БИП}

\par Этап заключается в преодолении некой области на земле. На земле сигнальной лентой создается сетка 6x6. Организатор имеет у себя <<карту>> этой сетки. На сетке есть безопасные участки и участки, на которые нельзя вступать. Существует маршрут по соседним клеткам каждая из которых безопасна из левой нижней в верхнюю правую. Клетки считаются соседними, если они соприкасаются стороной. По диагонали --- не соседи.

\par Задача команды \textbf{без слов, молча} найти это маршрут. Если участник во время <<прощупывания>> маршрута <<подорвался>>, то он возвращается к команде и \textbf{другой} участник команды продолжает поиск. В итоге, если маршрут найден все <<живые>> участники проходят по маршруту самостоятельно. <<Подорвавшихся>> переносят на спине <<живые>>.

\par Обо всех правилах заранее сообщается во вступительной речи \textbf{проводящего этап}.

\par \textbf{Речь для объяснения правил.} В разработке
%------------------------------



%------------------------------
\newsubsection{Крокодил}{Крокодил}

\par Команда садится в колонну друг за другом.

\par Задача --- передать впереди сидящего назад

\par Обо всех правилах заранее сообщается во вступительной речи \textbf{проводящего этап}.

\par \textbf{Речь для объяснения правил.} В разработке
%------------------------------



%------------------------------
\newsubsection{Электросеть}{Электросеть}

\par Между двумя деревьями создается сетка из сигнальной ленты с хаотично расположенными отверстиями.

\par Задача передать за сетки (на другую сторону) как можно больше членов команды. Запрещается касаться деревьев и самой сетки. После прохождения участника через отверстие в сетке, оно становится недоступным для дальнейшего использования. Если команда задевает сетки или дерево при передаче, результат аннулируется, этап начинается сначала. Над сеткой и под сеткой --- два разрешенных отверстия. Имеет смысл сразу сообщить об этом.

\par Обо всех правилах заранее сообщается во вступительной речи \textbf{проводящего этап}.

\par \textbf{Речь для объяснения правил.} В разработке
%------------------------------



%------------------------------
\newsubsection{Змейка}{Змейка}

\par Команда становится в колонну друг за другом. Руки на плечи впереди стоящего. Всем кроме первого закрыть глаза.

\par Задача преодолеть маршрут <<змейкой>>. Не подглядывая!

\par Обо всех правилах заранее сообщается во вступительной речи \textbf{проводящего этап}.

\par \textbf{Речь для объяснения правил.} В разработке
%------------------------------



%------------------------------
\newsubsection{Канаты}{Канаты}

\par Описание в разработке, ибо не видел конкурс в реальности. Жду повторных объяснений.

\par 

\par Обо всех правилах заранее сообщается во вступительной речи \textbf{проводящего этап}.

\par \textbf{Речь для объяснения правил.} В разработке
%------------------------------



%------------------------------
\newsubsection{Песня}{Песня}

\par Описание в разработке, ибо не видел конкурс в реальности. Жду повторных объяснений.

\par В прошлом году проводил Коля Пащенко, если я не ошибаюсь.

\par Обо всех правилах заранее сообщается во вступительной речи \textbf{проводящего этап}.

\par \textbf{Речь для объяснения правил.} В разработке
%------------------------------



%------------------------------
\newsubsection{(Воз)Душный шар --- (Пащенко Николай, 8-920-118-23-17)}{(Воз)Душный шар}

\par Описание в разработке. Конкурс предложен Николаем. Хочется описание составлять с ним.

\par 

\par Обо всех правилах заранее сообщается во вступительной речи \textbf{проводящего этап}.

\par \textbf{Речь для объяснения правил.} В разработке
%------------------------------



%------------------------------
\newsubsection{Сценарий}{Сценарий}

\par Как на игре на местности.

\par Описание в разработке. Сценарий, который будут изображать команды, в разработке.

\par Мария Аверина проводила этот этап во время игры на местности в 2015.

\par Обо всех правилах заранее сообщается во вступительной речи \textbf{проводящего этап}.

\par \textbf{Речь для объяснения правил.} В разработке
%------------------------------



%------------------------------
\newsubsection{Загадки}{Загадки}

\par Описание в разработке.

\par В прошлом году проводила Даша Петухова.

\par Обо всех правилах заранее сообщается во вступительной речи \textbf{проводящего этап}.

\par \textbf{Речь для объяснения правил.} В разработке
%------------------------------



%------------------------------
\newsubsection{Конкурс на меткость}{Конкурс на меткость}

\par Команде предлагается поразить мишень из лука. Каждому по 1-2 выстрела. Берем средний результат команды.

\par Необходимо создание большого количества стрел.

\par Обо всех правилах заранее сообщается во вступительной речи \textbf{проводящего этап}.

\par \textbf{Речь для объяснения правил.} В разработке
%------------------------------



%********************************************************************************





%********************************************************************************
\newsection{Спортивные мероприятия}{Спортивные мероприятия}

\par \textbf{Планируется ли проведение волейбола, капитошек, перетягивания каната?}
\par Спортивные этапы предвосхищаются речью главного распорядителя.



%------------------------------
\newsubsection{Турка --- (Иванов Константин)}{Турка}

Думаю, все в курсе, что это такое. Здесь необходимо указать требуемый инвентарь.
%------------------------------



%------------------------------
\newsubsection{Футбол}{Футбол}

Думаю, все в курсе, что это такое. Здесь необходимо указать требуемый инвентарь.
%------------------------------



%------------------------------
\newsubsection{Квиддич --- (Мосейкова Марина, 8-915-973-47-57)}{Квиддич}

\par Думаю все в курсе, что это такое. Здесь необходимо указать требуемый инвентарь.

\par Описание желательно составлять с идейным лидером данного мероприятия.

\par Описание правил тоже.
%------------------------------



%********************************************************************************





%********************************************************************************
\newsection{Ночная игра <<Голодные игры>>}{Ночная игра <<Голодные игры>>}

\par На каждый этап отводится 15 минут с учетом передвижения между точками. Сигналом к смене этапа служит <<выстрел пушки>>, в случае технических накладок --- SMS-рассылка кураторам, ведущим группы. Крайний случай --- таймеры у организаторов на этапе.
\par В разработке
%********************************************************************************





\end{document}