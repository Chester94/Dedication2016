\newsubsection{Квиддич}{Квиддич}

\subsubsection*{Необходимый инвентарь}
\begin{enumerate}
\item Мяч для американского футбола --- квофл
\item Два (2) мяча для волейбола --- бладжеры
\item Мяч для тенниса --- снитч
\item Опоры для колец
\item Кольца гимнастические 6 штук
\item Стяжки строительные
\item Сигнальная лента
\item Отличительные повязки на руки для команд
\end{enumerate}

\subsubsection*{Основная информация}
\begin{enumerate}
\item Состав для участия в игре на поле 7 человек: 1 вратарь, 3 охотника, 2 загонщика, 1 ловец.
\item Если команда не явилась к назначенному времени, то засчитывается поражение.
\item Если команда к началу игры явилась в неполном составе, то игры может быть начата без недостающих игроков по решению главного судьи.
\item За игру можно совершить 5 замен. Капитан сигнализирует главному судье, игрок передает повязку сменяющему игроку, игра не останавливается.
\item Команде необходимо выбрать капитана.
\item За нарушение следующих правил, команда не допускается к участию в данном мероприятии:
	\begin{itemize}
	\item Наличие острых предметов, длинных ногтей.
	\item Наличие предметов, способствующих получению игрового преимущества.
	\item Употребление нецензурной лексики.
	\item Подсказки и помощь участникам других команд.
	\end{itemize}
\end{enumerate}


\subsubsection*{Правила игры}
\begin{enumerate}
\item В матче принимают участие две команды по 7 игроков от каждой.
\item Целью игры является набрать максимальное количество очков за время игры. Команда получает десять (10) очков за каждое попадание в вороты, и пятьдесят (50) очков за ловлю Золотого Снитча.
\item Игра заканчивается, если одна из команд поймала Снитч, либо по истечении десяти минут игрового времени. Победителем считается команда, набравшая наибольшее количество очков.
\item Если в конце игры у обеих команд одинаковое количество очков, то дается дополнительное время.
\item Правильное положение метлы участника: метла находится между ног, не касаясь земли.
\end{enumerate}


\subsubsection*{Штрафы}
\begin{enumerate}
\item Нарушение 1-го типа:
	\begin{enumerate}
	\item Вернуться к своим воротам, держа метлу между ног, не касаясь других мячей и не влияя на ход игры.
	\item Обойти кольца, не заходя во вратарскую зону.
	\item Убедиться, что метла находится в нужном положении, написанном в данных правилах.
	\item Продолжить игру.
	\end{enumerate}
	
\item Нарушение 2-го типа:
	\begin{enumerate}
	\item Удалиться с поля игры, подойти к судье, отвечающему за штраф игрока.
	\item Минуту не принимать участия в игре.
	\item После разрешения судьи, вернуться на игровое поле.
	\end{enumerate}
\end{enumerate}


\subsubsection*{Вратарь}
\par\textbf{Цель игрока:} предотвратить попадание квофла (мяча для американского футбола) в одно из трех колец.
\par Игроку необходимо:
\begin{enumerate}
\item Находится в пределах вратарской зоны, покидать которую игроку запрещено. За нарушение данного правила команда получает штраф (с поля исключается один из охотников, который действует в соответствии с пунктом 2 штрафов).
\item Держать метлу в правильном положении. За нарушение данного правила команда получает штраф (с поля исключается один из охотников, который действует в соответствии с пунктом 2 штрафов).
\end{enumerate}

\par Игроку запрещено:
\begin{enumerate}
\item Участвовать в физических контактах с другими игроками. За нарушение данного правила игрок действует по пункту 2 штрафов.
\item Касаться преднамеренно снитча. За нарушение данного правила команда получает штраф (с поля исключается один из охотников, который действует в соответствии с пунктом 2).
\item Преднамеренно деформировать кольца. В случае изменения их конструкции незамедлительно сообщает об этом судьям.
\end{enumerate}
\par\textbf{Попадание бладжером не влияет на игрока}


\subsubsection*{Охотник}
\par\textbf{Цель игрока:} принести команде победу, забивая мяч в кольца противника.
\par Игроку необходимо:
\begin{enumerate}
\item Находится в пределах игрового поля, кроме вратарской зоны. Покидать игровое поле запрещено. За нарушение данного правила игрок действует по 1 пункту правил штрафа.
\item Держать метлу в правильном положении. За нарушение данного правила игрок действует по 1 пункту правил штрафа.
\end{enumerate}

\par Игроку разрешено:
\begin{enumerate}
\item Отнимать мяч у соперников.
\end{enumerate}

\par Игроку запрещено:
\begin{enumerate}
\item Захватывать игрока противника. За нарушение данного правила игрок действует по 2 пункту правил штрафа.
\item Преднамеренно касаться снитча. За нарушение данного правила игрок действует по 2 пункту правил штрафа.
\end{enumerate}

\par\textbf{Игрок, владеющий мячом, может совершить не более 5 шагов, после чего он требуется совершить бросок другому игроку. За нарушение данного правила необходимо действует в соответствии с 1 пунктом штрафов.}
\par\textbf{При попадании бладжером игрок действует в соответствии с пунктом 1 штрафов.}


\subsubsection*{Загонщик}
\par\textbf{Цель игрока:} нейтрализовать атакующий потенциал охотников противника, используя бладжеры.
\par Игроку необходимо:
\begin{enumerate}
\item Перемещаться по всему игровому полю, кроме вратарской зоны.
\item Держать метлу в правильном положении. За нарушение данного правила игрок действует по 1 пункту правил штрафа.
\end{enumerate}

\par Игроку разрешено:
\begin{enumerate}
\item Выходить за пределы поля (но только для того, чтобы вернуть мяч в игру). Кидать мяч разрешено только в игровой зоне, в игроков, в ней находящихся.
\item Попадать бладжером (волейбольным мячом) в игроков соперника. Разрешается только бросать мяч, нельзя ударить им игрока, и продолжать держать его в руках.
\item Совершить перехват бладжера: при условии, что загонщик поймал бладжер, до падения бладжера на землю. Таким образом бладжер переходит к нему и он может атаковать игрока соперника.
\item Охранять игроков своей команды, а так же самого себя отражая броски загонщиков противника бладжером, который сам охотник держит в руках.
\end{enumerate}

\par Игроку запрещается:
\begin{enumerate}
\item Брать в руки квоффл, а также предумышленно изменять траекторию его движения.
\item Захватывать игрока противника. За нарушение данного правила игрок действует по 2 пункту правил штрафа.
\item Преднамеренно касаться снитча. За нарушение данного правила игрок действует по 2 пункту правил штрафа.
\end{enumerate}

\par\textbf{Если бладжер коснулся земли, то любое касание игроком этого мяча. игнорируется игроком. Однако любым игрокам, кроме загонщиков, запрещается совершать предумышленное касание лежащего на земле бладжера.}
\par\textbf{При попадании бладжером игрок действует в соответствии с пунктом 1 штрафов.}


\subsubsection*{Ловец}
\par\textbf{Цель игрока:} поймать снитч при счете, который устраивает его команду. До появления снитча ловец может принимать на себя удары бладжером, оберегая охотников.
\par Игроку необходимо:
\begin{enumerate}
\item Перемещаться по всему игровому полю, кроме вратарской зоны.
\item Держать метлу в правильном положении. За нарушение данного правила игрок действует по 1 пункту правил штрафа.
\end{enumerate}

\par Игроку разрешено:
\begin{enumerate}
\item Ловить снитч до касания им земли.
\end{enumerate}

\par Игроку запрещено:
\begin{enumerate}
\item Брать в руки квоффл, а также предумышленно изменять траекторию его движения.
\item Захватывать игрока противника. За нарушение данного правила игрок действует по 2 пункту правил штрафа.
\end{enumerate}

\par\textbf{При попадании бладжером игрок действует в соответствии с пунктом 1 штрафов.}