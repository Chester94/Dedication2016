\newsection{Основная сюжетная линия}{Основная сюжетная линия}



%********************************************************************************
\newsubsection{Введение}{Введение}

\par В качестве сюжета посвята была выбрана тематика романа Сьюзен Коллинз <<Голодные игры>>: первая часть трилогии, а также финал романа.

\par Организаторы (профбюро) выступают в роли Капитолия и его жителей, группы первокурсников --- в роли трибутов из дистриктов.

\par Каждый из 12(13) фигурирующих в произведении дистриктов (районов), специализируется на конкретной отрасли промышленности:

\begin{enumerate}
\item предметы роскоши для Капитолия;
\item изначально каменщики, затем база подготовки миротворцев;
\item производство электроники, автомобилей, оружия;
\item рыболовство;
\item энергетика;
\item производство транспортных средств (мало информации в целом);
\item пиломатериалы, древесина, бумага;
\item текстиль, униформа для военных;
\item сельское хозяйство;
\item животноводство;
\item сельское хозяйство (в отличии от 9, возделываются различные культуры, в том числе хлопок);
\item угольная промышленность;
\item официально графит, реально --- ядерное оружие.
\end{enumerate}

\par Лично я считаю, что дистрикт 6 не стоит выбирать, так как информации о нем недостаточно, кроме того он полностью может быть перекрыт 3. По схожим соображениям дистрикт 9 стоит сразу же исключить из списка, так как он в полной мере может быть заменен 11. Использование дистрикта 13 стоит вынести на обсуждение из-за его обособленности в сюжете романа и привилегированного положения в целом. Также стоит уточнить, что данный дистрикт не принимал участия в <<Голодных играх>>. Не исключаю возможности использования этого дистрикта и в сюжете посвята в качестве некой организации в стане Капитолия, которая способствовала предполагаемой революции. Итого на данный момент считаю, что в наличии есть 10 дистриктов, из которых необходимо выбрать как \textbf{минимум 5} наиболее простых для реализации первокурсниками и одновременно как можно более интересных с точки зрения отыгрыша и сюжета в целом.

\par Начиная с 1 сентрября 2016 года, осуществляется подготовка первокурсников к данному мероприятию. Учитывая вышесказанное, организаторам необходимо заранее определить список наиболее подходящих дистриктов. Именно из этого списка первокурсники случайным образом выберут себе тематику группы. Жеребьевку имеет смысл проводить 2 сентября, так как это пятница, что позволит дать людям хотя бы одни полные выходные.

\par К моменту проведения жеребьевки было бы неплохо кратко, не вдаваясь в подробности, ознакомить первокурсников с тематикой посвята. Закрепление дистрикта за группой планируется следующим образом. Первокурсникам заранее объявляется весь набор возможных дистриктов. В <<шляпу>> кладутся листочки с номерами и кратким описанием отрасли промышленности. Затем группа выдвигает своего представителя, чтобы он вытянул их судьбу. Если группу устраивает выпавшее значение и тематика, то выбор для данной группы окончен, они могут быть свободны. В случае, если группа не согласна с выбором, то представитель группы кладет листок обратно в <<шляпу>> и отправляется в конец очереди. Изначально очередь формируется следующим образом ПМИ~--~11, ПМИ~--~12, ФИИТ~--~11, ФИИТ~--~12, ПИЭ~--~11. Если я правильно помню, то порядок в очереди никак не влияет на шансы получить тот или иной дистрикт, поэтому очередность чисто формальная.

\par После завершения процедуры выбора тематик группы могут приступать к подготовке к посвяту. Название команды, девиз, внешний вид и визитка команды должны соответствовать выбранной тематике, в разумных пределах, конечно. Кураторам необходимо следить и направлять группы с учетом этого пожелания, насколько это возможно в принципе.
%********************************************************************************



%********************************************************************************
\newsubsection{Концепция трех ролей}{Концепция трех ролей}

\par Среди организаторов выделяются 3 роли, исполнители которые более других должны соответствовать своим ролям\footnote{см. раздел <<Роли>>}:
\begin{enumerate}
\item Президент Сноу;
\item Главный распорядитель игр;
\item Цезарь Фликерман (журналист, репортер).
\end{enumerate}

\par Кроме того планируется введение четвертой эпизодической роли --- Предводитель повстанцев (революционеров). Именно он скажет, что во время <<Голодных игр>>, пока вся страна наблюдала за кровавой резней, группа из бывших победителей -- менторов, кураторов -- свергла власть и захватила в плен Сноу и его приближенных.
%********************************************************************************



%********************************************************************************
\newsubsection{Краткое содержание сюжета}{Краткое содержание сюжета}

\par Около 11-00 первокурсники централизовано автобусами доставляются к месту проведения мероприятия\footnote{см. раздел <<Расписание>>}. После вступительного слова профорга и иных организационных мероприятий организаторам желательно начать отыгрывать роль жителей Капитолия (в разумных пределах).

\par После установки первокурсниками лагерей следует краткая речь президента\footnote{речь президента и распорядителя смотри в разделе <<Роли>>}. Затем --- речь распорядителя игр.

\par После завершения <<официальной части>> следуют визитки участников. После каждой визитки репортер (Цезарь Фликерман) задает группе, которая еще не ушла со <<сцены>>, несколько (около 3) вопросов\footnote{вопросы для репортера смотри в разделе <<Роли>>}. Фактически Цезарь полностью проводит визитки: объявление, представление команд, общий конферанс.

\par Затем распорядитель игр объявляет о начале тренировок перед играми, куда входят дневные этапы\footnote{см. раздел <<Дневные этапы>>}, они же --- <<Вертушка>>.

\par Каждую команду первокурсников ведет по этапам <<Вертушки>> один из их кураторов (менторов). Маршрутные листы будут выданы отдельно\footnote{полные правила изложены в разделе <<Дневные этапы>>}.

\par После <<Вертушки>> следует обед, затем турка, футбол и квиддич. Обоснование и мотивацию проведения данных мероприятий участникам объясняет главный распорядитель игр.

\par Затем ужин и непосредственно <<Голодные игры>> в виде Ночной игры.

\par События ночной игры выводят сюжет на революцию. После прибытия всех первокурсников на место финального костра следует финал сюжета. Лидер повстанцев, захвативших власть и свергнувших президента, выступает перед группами и рассказывает о событиях, <<происходивших>> параллельно с <<Голодными играми>>. Он же объявляет и финал сюжета.

\par На данный момент существует два основных варианта завершения истории. Первый, наиболее приближенный к оригиналу, --- казнь президента (также возможно Главного распорядителя и Цезаря Фликермана), каким образом будет осуществляться казнь --- на данный момент не решено. Второй вариант более мирный, но предполагает некий юмористический подтекст. Лидер повстанцев объявляет лишь о некоторых наказаниях для виновных или несчастных случаях, которые могли с ними произойти. Например, президент Сноу приговаривается к \textit{пожизненному изучению матанализа}, а Главный распорядитель игр \textit{<<случайно>> отравился макаронами, сваренными два месяца назад студентами третьего курса, проживающими в общежитии}. Цезаря Фликермана имеет смысл <<помиловать>>, как лицо наименее <<виновное>>.

\par Команде, набравшей наибольшее количество очков по результатам всего мероприятия, выдается приз. В качестве приза предлагается 4 бутылки шампанского или право казнить Сноу (если будет выбран первый вариант сценария). Команда может и отказаться от роли палачей. Этот момент не стоит исключать, необходимо отдельно его продумать, хоть я и считаю его маловероятным. Кто же откажется казнить Сноу? Вопрос о награждении победителей считаю открытым и требующим отдельного обсуждения расширенным собранием.
%********************************************************************************