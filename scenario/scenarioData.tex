\newsection{Основная сюжетная линия}{Основная сюжетная линия}



%********************************************************************************
\newsubsection{Введение}{Введение}

\par В качестве сюжета посвята была выбрана тематика романа Сьюзен Коллинз <<Голодные игры>>: первая часть трилогии, а также финал романа.

\par Организаторы (профбюро) выступают в роли Капитолия и его жителей, группы первокурсников --- в роли трибутов из дистриктов.

\par Каждый из 12(13) фигурирующих в произведении дистриктов (районов), специализируется на конкретной отрасли промышленности. Из имеющихся выбрано 10 дистриктов:

\begin{itemize}
\item \textbf{Дистрикт №1} предметы роскоши для Капитолия;
\item \textbf{Дистрикт №2} изначально каменщики, затем база подготовки миротворцев;
\item \textbf{Дистрикт №3} производство электроники, автомобилей, оружия;
\item \textbf{Дистрикт №4} рыболовство;
\item \textbf{Дистрикт №5} энергетика;
\item \textbf{Дистрикт №7} пиломатериалы, древесина, бумага;
\item \textbf{Дистрикт №8} текстиль, униформа для военных;
\item \textbf{Дистрикт №9} животноводство;
\item \textbf{Дистрикт №11} сельское хозяйство;
\item \textbf{Дистрикт №12} угольная промышленность;
\end{itemize}

\par Начиная с 1 сентрября 2016 года, осуществляется подготовка первокурсников к данному мероприятию. Учитывая вышесказанное, организаторам необходимо заранее определить список наиболее подходящих дистриктов. Именно из этого списка первокурсники случайным образом выберут себе тематику группы. Жеребьевку имеет смысл проводить 2 сентября, так как это пятница, что позволит дать людям хотя бы одни полные выходные.

\par К моменту проведения жеребьевки было бы неплохо кратко, не вдаваясь в подробности, ознакомить первокурсников с тематикой посвята. Закрепление дистрикта за группой планируется следующим образом. Первокурсникам заранее объявляется весь набор возможных дистриктов. В <<шляпу>> кладутся листочки с номерами и кратким описанием отрасли промышленности. Затем группа выдвигает своего представителя, чтобы он вытянул их судьбу. Если группу устраивает выпавшее значение и тематика, то выбор для данной группы окончен, они могут быть свободны. В случае, если группа не согласна с выбором, то представитель группы кладет листок обратно в <<шляпу>> и отправляется в конец очереди. Изначально очередь формируется следующим образом ПМИ~--~11, ПМИ~--~12, ФИИТ~--~11, ФИИТ~--~12, ПИЭ~--~11. Если я правильно помню, то порядок в очереди никак не влияет на шансы получить тот или иной дистрикт, поэтому очередность чисто формальная.

\par После завершения процедуры выбора тематик группы могут приступать к подготовке к посвяту. Название команды, девиз, внешний вид и визитка команды должны соответствовать выбранной тематике, в разумных пределах, конечно. Кураторам необходимо следить и направлять группы с учетом этого пожелания, насколько это возможно в принципе.
%********************************************************************************



%********************************************************************************
\newsubsection{Концепция трех ролей}{Концепция трех ролей}

\par Среди организаторов выделяются 3 роли, исполнители которые более других должны соответствовать своим ролям\footnote{см. раздел <<Роли>>}:
\begin{enumerate}
\item Президент Сноу;
\item Главный распорядитель игр;
\item Цезарь Фликерман (журналист, репортер).
\end{enumerate}

\par Кроме того планируется четвертая эпизодическая роль --- Предводитель повстанцев (революционеров). Именно он скажет, что во время <<Голодных игр>>, пока вся страна наблюдала за кровавой резней, группа из бывших победителей -- менторов, кураторов -- свергла власть и захватила в плен Сноу и его приближенных.
%********************************************************************************



%********************************************************************************
\newsubsection{Краткое содержание сюжета}{Краткое содержание сюжета}

\par Около 11-00 первокурсники централизовано автобусами доставляются к месту проведения мероприятия\footnote{см. раздел <<Расписание>>}. После вступительного слова профорга и иных организационных мероприятий организаторам желательно начать отыгрывать роль жителей Капитолия (в разумных пределах).

\par После установки первокурсниками лагерей следует краткая речь президента\footnote{речь президента и распорядителя смотри в разделе <<Роли>>}. Затем --- речь распорядителя игр.

\par После завершения <<официальной части>> следуют визитки участников. После каждой визитки репортер (Цезарь Фликерман) задает группе, которая еще не ушла со <<сцены>>, несколько (около 3) вопросов\footnote{вопросы для репортера смотри в разделе <<Роли>>}. Фактически Цезарь полностью проводит визитки: объявление, представление команд, общий конферанс.

\par Затем распорядитель игр объявляет о начале тренировок перед играми, куда входят дневные этапы\footnote{см. раздел <<Дневные этапы>>}, они же --- <<Вертушка>>.

\par Каждую команду первокурсников ведет по этапам <<Вертушки>> один из их кураторов (менторов). Маршрутные листы будут выданы отдельно\footnote{полные правила изложены в разделе <<Дневные этапы>>}.

\par После <<Вертушки>> следует обед, затем турка, футбол и квиддич. Обоснование и мотивацию проведения данных мероприятий участникам объясняет главный распорядитель игр.

\par Затем ужин и непосредственно <<Голодные игры>> в виде Ночной игры.

\par События ночной игры выводят сюжет на революцию. После прибытия всех первокурсников на место финального костра следует финал сюжета. Лидер повстанцев, захвативших власть и свергнувших президента, выступает перед группами и рассказывает о событиях, <<происходивших>> параллельно с <<Голодными играми>>. Он же объявляет и финал сюжета.

\par В качестве финала предлагается публичная казнь Кориолануса Сноу через повешение. Победившая команда получает право решить судьбу президента. Они могут его помиловать. Главный распорядитель и Цезарь Фликерман вообще никак не <<наказываются>>.
%********************************************************************************